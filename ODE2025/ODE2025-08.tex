\section{Bernoulliの方程式・Riccatiの方程式} %第8回
\subsection{Bernoulliの方程式}
次のような方程式を考えよう:
\begin{equation}
    y' + P(x)y = Q(x)y^{\alpha}. \label{eq:ast-ber}
\end{equation}
$\alpha=0$ならば,\eqref{eq:ast-ber}は非同次線形方程式である.一方$\alpha = 1$ならば$y'+(P(x)-Q(x))y=0$となり,これは同次線形方程式である.以後,$\alpha \neq 0, 1$としよう.

ここで,$\bm{z(x)=y(x)^{1-\alpha}}$と変換すると$z'(x)=(1-\alpha)y^{-\alpha}y'$であるから,\eqref{eq:ast-ber}について両辺$y^{\alpha}$で割ると
\[ \frac{y'}{y^{\alpha}} + P(x)y^{1-\alpha} = Q(x)\]
となる.したがって
\begin{align*}
    \frac{z'(x)}{1-\alpha} + P(x)z(x) &= Q(x) \\
    \ie \quad z'(x) + (1-\alpha)P(x)z(x) &= (1-\alpha)Q(x).
\end{align*}
よって,非同次の線形方程式となるから,これは解ける($\alpha=0$のときに帰着).

\begin{example} \label{example:ber}
    $\displaystyle y'+\frac{y}{x} = x^2y^3$を解く.$P(x) = 1/x, \; Q(x) = x^2, \; \alpha=3$といえる.では,$z(x) \coloneqq y{1-3} = y^{-2}$となる.$z'=-2y^{-3}y'$であり,考えている方程式について両辺$y^3$で割ると
    \begin{align*}
        \frac{y'}{y^3} + \frac{1}{x}y^{-2} &= x^2 \\
        \quad -\frac{1}{2}z' + \frac{1}{x}z &= x^2 \\
        \ie \quad z'-\frac{z}{x} &= -2x^2 \tag{eq:ast-section8}
    \end{align*}
    を得る.ではこれを解こう.$z'-z/x= 0$の解を求めてから\eqref{eq:ast-section8}の解を求めようとするとしんどいと思われるので,適当な積分因子をかけることにより面倒さ(?)を回避しよう.

    両辺に$x^{-2}$をかけると,以下のようになる:
    \begin{align*}
        \frac{1}{x^2}z' - \frac{2}{x^3}z &= -2 \\
        \frac{d}{dx}\left(\frac{1}{x^2}z\right) &= -2 \\
    \end{align*}
    より
    \begin{align*}
        \frac{1}{x^2}z &= -2x + c \\
        z &= -2x^3+cx^2 \\
        \frac{1}{y^2} &=  -2x^3+cx^2 \\
        \therefore \quad y^2(-2x^3+cx^2) &= 1.
    \end{align*}
    ちなみに,$y \equiv 0$も解である.
\end{example}

\begin{homework*}[任意]
    次の手順で例\ref{example:ber}を解け.
    \begin{enumerate}
        \item $z'-z/x= 0$の解を求めよ.
        \item $z'-z/x = -2x^2$の解を求めよ.
    \end{enumerate}
\end{homework*}

\begin{remark}
    Bernoulliの方程式において,$\alpha>0$のときは$y'+P(x)y=Q(x)y^{\alpha}$について,$y \equiv 0$も解となっている.
\end{remark}

\subsection{Riccatiの方程式}
次のような方程式を考えよう:
\begin{equation}
    y'(x) + P(x) + Q(x)y(x) + R(x)y^2 = 0. \label{eq:riccati}
\end{equation}
$R(x) \equiv 0$ならば,\eqref{eq:riccati}は線形微分方程式である.一方,$P(x) \equiv 0$ならば,$\alpha=2$のBernoulli型である.

さて,\eqref{eq:riccati}は一般には解きにくい.幸いなことに,1つの解$y_0$がわかると,一般解を記述できる.

$y_0$を\eqref{eq:riccati}の解とすると,次がなりたつ:
\[ {y_0}' + P(x) + Q(x)y_0 + R(x){y_0}^2 = 0. \label{eq:riccati-y_0} \]
\eqref{eq:riccati}$-$\eqref{eq:riccati-y_0}を実行すると
\begin{align*}
    (y-y_0)' + (P(x)-P(x)) + Q(x)(y-y_0) + R(x)(y^2-{y_0}^2) &= 0 \\
    \ie \quad (y-y_0)' + Q(x)(y-y_0) + R(x)(y^2-{y_0}^2) &= 0.
\end{align*}
$y^2-{y_0}^2$について,こいつを上手く変形すれば,Bernoulliの方程式に帰着できそうである.実際,次のように変形すれば達成される:
\begin{align*}
    y^2-{y_0}^2 &= (y-y_0)^2 + 2yy_0 - {y_0}^2-y^2 \\
    &= (y-y_0)^2 +  2y_0(y-y_0).
\end{align*}
$u(x) \coloneqq y-y_0$とおくと
\begin{align*}
    u'(x) + Q(x)u(x) + R(x)(u^2(x)+2y_0u(x)) &= 0 \\
    \therefore \quad u'(x) + (Q(x)+2y_0)u(x) + R(x)u^2(x) &= 0.
\end{align*}
これはBernoulli型の方程式!

\begin{example}
    $y'+(2x-1)y-(x-1)y^2=x$を解く.まず$y \equiv 1$が解であることがわかる(直接代入して確認せよ).
    
    $u(x) \coloneqq y-1$とおくと$y=u(x)+1$.$y'=u'$と合わせて次を得る:
    \begin{align*}
        u'(x) + (2x-1)(u+1) - (x-1)(u'+1) &= x \\
        \ie \quad u'(x) L+ u(x) = (x-1)u^2.
    \end{align*}
    これは$\alpha=2$のBernoulli方程式であり,$\alpha>0$より,$y \equiv 0$は解の1つである.

    $u'(x) L+ u(x) = (x-1)u^2$を解こう.$z \coloneqq u^{1-2}=u^{-1}$とおく.$z'=-u^{-2}u'$であり,両辺$u^{-2}$をかけると
    \begin{align*}
        \frac{u'}{u^2} + u^{-1} &= x-1 \\
        \therefore \quad -z' + z &= x-1.
    \end{align*}
    これ以降は各自の宿題とする(授業では演習として各自が解いた記憶がある.なのでみんなも計算しよう).
\end{example}

\begin{homework*}
    好きな方法でよいから,$u'(x) + u(x) = (x-1)u^2$を解け.

    答えは$y=\frac{1}{xe^{-x}+c}+1, \; y \equiv 0, \; 1$である.
\end{homework*}