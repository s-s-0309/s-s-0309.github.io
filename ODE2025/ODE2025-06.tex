\section{同次形とその具体例} %第6回
\subsection{同次形}
\begin{example}
    $y'=\frac{xy}{x^2+y^2}$を解く.

    $\frac{xy}{x^2+y^2}=\frac{y/x}{1+(y/x)^2}$より,これは同次形である.$z(x) \coloneqq y/x$とおく.$y=zx$より$y'=z+xz'$であるから
    \begin{align*}
        z+xz' &= \frac{z}{1+z^2} \\
        \ie \quad xz &= \frac{-z^3}{1+z^2}.
    \end{align*}
    したがって以下を得る:
    \[ z' = -\frac{1}{x} \cdot \frac{z^3}{1+z^2}.\]
    これは$P(x)=-1/x, \; Q(z)=\frac{z^3}{1+z^2}$変数分離形であるから,前回学習した方法で解を求めることができる.

    (1) $Q(z)=0$のとき,$z=0$である(これを\textbf{自明解}という).

    (2) $Q(z) \neq 0$のとき,両辺を$Q(z)$で割ると
    \begin{align*}
        \int \frac{1+z^2}{z^3} \; dz &= \int -\frac{1}{x} \; dx \textcolor{blue}{+c} \\
        \int \left(-\frac{1}{z^3} + \frac{1}{z}\right) \; dz &= -\log |x| + c \\
        \ie \quad -\frac{1}{2z^3} + \frac{1}{z} &= -\log |x| + c.
    \end{align*}

    (3) $y$と$x$の式に直そう.
    \begin{align*}
        -\frac{1}{2}\left(\frac{x}{y}\right)^2 + \log \left|\frac{y}{x}\right| &= -\log |x| + c \\
        -\frac{1}{2}\left(\frac{x}{y}\right)^2 + \log |y| - \log |x| &= -\log |x| + c \\
        \ie \quad -\frac{1}{2}\left(\frac{x}{y}\right)^2 + \log |y| &= c.
    \end{align*}

    よって,求める解は$y=0, \; -\frac{1}{2}\left(\frac{x}{y}\right)^2+\log |y| =c$である.
\end{example}

$P,Q$が多項式であり,$f\left(y/x\right) = Q(x,y)/P(x,y)$とかけているとき,次のことがなりたつ:
\begin{claim*}
    同次形であるためには,$P,Q$のすべての項が$x,y$について同じ次数$k$であればよい.
\end{claim*}

\begin{example}[$k=2$]
    $k=2$として,$P(x,y)=ax^2+bxy+cy^2, \; Q(x,y)=\alpha x^2 + \beta xy + \gamma y^2$のとき,分子分母を$x^k$で割ると
    \[ \frac{Q(x,y)}{P(x,y)} = \frac{\alpha + \beta (\frac{y}{x}) + \gamma (\frac{y}{x})^2}{a + b (\frac{y}{x}) + c (\frac{y}{x})^2}\]
    となる.
\end{example}

\begin{example} \label{example:homework}
    $y'=\frac{x-2y}{2x+y}$を解く.

    分子分母を$x$で割ると
    \[ y' = \frac{1-2\frac{y}{x}}{2+\frac{y}{x}} = \frac{1-2z}{2z+1}.\]
    となる.ここで$z \coloneqq y/x$とおいた.$y=xz$について両辺微分すると$y'=z+xz'$となるから
    \begin{align*}
        z+xz' &= \frac{1-2z}{2z+1} \\
        \ie \quad z' &= \frac{1}{x} \cdot \frac{-1-3z}{2+z}.
    \end{align*}

    さて,これは$P(x)=1/x, \; Q(z) = (-1-3z)/(2+z)$の変数分離形である.これを解くのは各自の宿題としよう\footnote{実際には講義中で演習として計算したが,この講義ノートは\LaTeX でつくられている故打ち込むのが面倒だから,宿題としている.}
\end{example}

\begin{homework*}
    例\ref{example:homework}を解け.
    
    ヒント:途中で$\int \frac{z+2}{z^2+4z-1} \; dz$の積分が出てきてこれに戸惑うかもしれない.これは次のように変形するとうまくいく:
    \[ \int \frac{z+2}{z^2+4z-1} \; dz = \int \frac{z+2}{(z+2)^2-5} \; dz = \frac{1}{2}\log |(z+2)^2-5|. \]

    解は$y/x=-2\pm \sqrt{5}, \; \frac{1}{2}\log|(y/x)^2+4(y/x)-1|=- \log|x| + c$となる.

    もっと簡潔に表したい場合は頑張って計算すること.最終的には$y^2+4xy-x^2=\tilde{c}$に落ち着く.
\end{homework*}

\subsection{同次形の方程式が必要か?(具体例は?)}
図があって書くのが大変なので省略.ここよりも同次形の解き方をマスターすることが大事.

\subsection{同次形に帰着できる問題}
先ほどの例でやったように,常微分方程式$y'=\frac{x-2y}{2x+y}$は同次形である.さて,これが$y'=\frac{x-2y+1}{2x+y+b}$といったように,分子分母に定数項が加わるとどうなるだろうか.これは同次形ではなくなる.ではどのようにして解けばよいか.それはある変数変換を行うことにより,解くことが可能になる.

$x-2y+1=0, \; 2x+y+b=0$をみたす$(x,y)$を$(\alpha,\beta)$と書くことにする.ここで,
\[ X \coloneqq x-\alpha, \; Y \coloneqq y-\beta\]
という変数変換を施す.すると,以下の計算結果が得られる:
\begin{align*}
    x-2y+a &= X-2Y, \\
    2x+y+b &= 2Y+X.
\end{align*}
$dy/dx = dY/dX$であるから
\[ \frac{dY}{dX} = \frac{X-2Y}{2Y+X}\]
を得る.\textbf{これは同次形である!}

ちなみに,$x-2y+1=0, \; 2x+y+b=0$をみたす$(x,y)$を求めるには,以下をとけばよい.これがピンとこない場合は,線形代数について復習すること.
\[ \begin{pmatrix}
    1 & -2 \\ 2 & 1
\end{pmatrix}
\begin{pmatrix}
    x \\ y
\end{pmatrix}
= \begin{pmatrix}
    a \\ b
\end{pmatrix}.\]

\begin{example}
    $\frac{dy}{dx} = \frac{4x+y-4}{x+y-1}$を解く.

    まず,次の連立方程式
    \[ \begin{cases}
        4x+y-1 = 0 \\
        x+y-1 = 0
    \end{cases}
    \iff
    \begin{pmatrix}
        4 & 1 \\
        1 & 1
    \end{pmatrix}
    \begin{pmatrix}
        x \\ y
    \end{pmatrix}
    = 
    \begin{pmatrix}
        1 \\ 1
    \end{pmatrix}\]
    を解くと,$(x,y)=(1,0)$を得る.

    ここで,$u \coloneqq x-1, \; v \coloneqq y-0$とおく.$x=u+1, \; y=v$であるから
    \[ \frac{dy}{dx} = \frac{dv}{du} = \frac{4u+v}{u+v}\]
    となり,これは同次形である.あとはガリガリ計算すると,以下を得る:
    \begin{align*}
        (y-2x+2)^3(y+2x+2) = c \quad (c \neq 0).
    \end{align*}
\end{example}