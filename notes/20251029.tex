\documentclass[dvipdfmx,11pt]{jsarticle}
\usepackage{/Users/shosaitou/Documents/TeXShop/tex-template/mystyle}

\begin{document}
\textbf{宿題(10月28日)}
\begin{enumerate}[label=宿題\arabic*: ]
    \item $\mathbb{N},\mathbb{Q},\mathbb{Q} \subset \mathbb{R}$と考えるとうまく行きそうなのはなんでだろう.
    \item 自然数の構成方法について調べよう$\to$ペアノの公理を調べる.
\end{enumerate}

\noindent \hrulefill

\subsection*{コメント}
\subsection{$a\leq 0 \implies -a \geq 0$について(笠井先生)}
よく「移項する」とか「両辺$(-1)$倍する」とかいう.これらの本質(これらの結果が導かれる所以)が何処にあるかと言えば「両辺xx加えても,結果は変わらない(不等号の向きが保たれる)」という性質である.

中学生に説明するときは,次が説明されるとよい.

\subsection{ローマ数字について(中川先生)}
(i),(ii)は「かっこいち,かっこに」とよむ.「i(あい)」とか読んでしまいそうだが,これはローマ数字である.

ちなみに,2025をローマ数字でかくと"MMXXV"である.

\subsection{命題2.1(ii)について(笠井先生)}
命題2.1(ii)は「$a$と$b$の間にある数を1つあげよう」という問いに対する1つの回答として,素朴に使える一番シンプルな例である.したがって,このような問いが出されたときは,まず命題2.1(ii)を試してみよう.

\subsection{$54!$はどれくらい?(中川先生)}
これは大体67桁くらいになるそうだ.推定の仕方としては,$54! > 10^54$を使うといったように,不等式評価をする.常用対数を用いてもよい.

ちなみに,スターリングの公式を使って推定しても良い(これは統計力学でも使われるらしい).

\begin{theorem}[スターリングの公式,(cf: 杉浦解析1, p340.)]
    \[ n! \sim  \sqrt{2\pi} n^{n+1/2} e^{-n}, \quad (n \to +\infty).\]
\end{theorem}
$n$がわりかし大きいときに精度がいい.数値実験は,余裕があれば...

\subsection{数直線の説明について(笠井先生)}
中学生に数学を教える場合は,避けて通れない.自分ならどう説明するかを考えておくこと.

\subsection{数直線の説明で$OA=a \cdot OE$とかいたけど...(中川先生)}
Eというのは,実数1に対応づけられた点のことである(教科書参照).さて,Aの位置はOの右側になることは確定しているが,Eに関して左右どちらにくるかはまだわからない.

$0 < a < 1$であれば,OとEの間にAは来る.一方,$a>1$であれば,Eの右側に来る.

さて,次のような質問にはどう答えようか:\\
\centerline{「$OA=\frac{1}{a}OE, 0 < a$」なるとき,Aはどこにあるか} \\
Aは,Eを含まずEの右側に来る.おそらく,複素数平面などで取り扱われる「反転」のことを言っているのだと思う.

\end{document}