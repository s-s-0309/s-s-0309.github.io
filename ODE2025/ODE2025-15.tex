\section{行列の指数関数(その2)}
念のため,前回までの復習をしよう.\eqref{eq:14-I}の解は,次となることを確かめた:
    \[ \bm{y}(x) = e^{xA}\bm{\alpha} + \int_{0}^{x} e^{(x-t)A}\bm{b}(t) \; dt.\]
前回を通して一貫した疑問に「$e^{xA}$は何者なのか」が挙げられる.今回はこれについて答えを与えよう.

以後の議論では,
\begin{center}
    ($\ast$) 行列$A$に対して,ある正則行列$P$があって$P^{-1}AP$が「簡単」になるとき
\end{center}
を考えることとする.

\begin{note*}[「簡単」とは]
    「簡単」とは,$A$の固有値を求めたとき,各固有値に対する固有ベクトルが合わせて$n$個あることを指す.

    ちなみに,$P^{-1}AP$が対角行列にならない場合がある.いわゆる\textbf{ジョルダン標準形}になるときである.これも含めた議論についてはいろいろな本に書いてある.わたしは\ref{book:takahashi}でそれをみたとき,そっと本を閉じた.読む気にもなれなかった.ただ一度は読まねばならない.
\end{note*}
\begin{talk*}
    中川先生も,ジョルダン標準形も含めた議論を初めて見たとき,嫌になって本を閉じたらしい.
\end{talk*}

($\ast$)の意味について考察しよう.$A$の固有値$\lambda_1 \leq \lambda_2 \leq \cdots \leq \lambda_n$に対して,固有ベクトルをそれぞれ$\bm{u}_1,\; \bm{u}_2, \ldots, \bm{u}_n$とおく.正則行列$P$を次で定めよう:
\[ P \coloneqq (\bm{u}_1 \; \bm{u}_2 \; \cdots \; \bm{u}_n)^\top. \]
ただし,$P$は実$n \times n$正方行列である.このとき,$P$は正則で逆行列$P^{-1}$が存在する.
したがって
\[ B = P^{-1}AP = 
\begin{pmatrix}
    \lambda_1 & & & \text{\Large O} \\
     & \lambda_2 & & \\
     & & \ddots & \\
     \text{\Large O}& & & \lambda_n
\end{pmatrix}\]
となる.対角成分以外は0であることに注意しよう(これは($\ast$)という場合を考えているから).

さて,行列の指数関数$e^{xA}$は次で定義されていた:
\[ e^{xA} \coloneqq 1 + xA + \frac{1}{2!}(xA)^2 + \cdots + \frac{1}{n!}(xA)^n + \cdots. \]
$A^n$をみるために,$B=P^{-1}AP$を用いる.$A=PBP^{-1}$より,$A^n=(PBP^{-1})^n$である.
\begin{align*}
    A^n &= (PBP^{-1})(PBP^{-1}) \cdots (PBP^{-1}) \\
    &= PB^nP^{-1}.
\end{align*}
よって,$A^n=PB^nP^{-1}$を用いて
\begin{align*}
    e^{xA} &= PP^{-1} + xPBP^{-1} + \frac{x^2}{2!}(PB^2P^{-1}) + \cdots + \frac{1}{n!}(PB^nP^{-1}) + \cdots \\
    &= P(I + xB + \frac{x^2}{2!}B^2 + \cdots + \frac{x^n}{n!}B^n + \cdots)P^{-1}
\end{align*}
とできる.以上より,$e^{xA}=Pe^{xB}P^{-1}$を得る.

$e^{xB}$に着目しよう.
\begin{align*}
    e^{xB} &= 
    \begin{pmatrix}
        1+x\lambda_1+\frac{(x\lambda_1)^2}{2!}+\cdots & 0 & \cdots & 0 \\
        0 & 1+x\lambda_2+\frac{(x\lambda_2)^2}{2!}+\cdots & \cdots & 0 \\
        \vdots & \vdots& \ddots & \vdots \\
        0 & 0 & \cdots & 1+x\lambda_n+\frac{(x\lambda_n)^2}{2!}+\cdots
    \end{pmatrix} \\
    &= 
    \begin{pmatrix}
        e^{\lambda_1x} & 0 & \cdots & 0 \\
        0 & e^{\lambda_2x} & \cdots & 0 \\
        \vdots & \vdots & \ddots & \vdots \\
        0 & 0 & \cdots & e^{\lambda_nx}
    \end{pmatrix}
\end{align*}
より,$e^{xB}$が対角行列として表現できたから,$e^{xA} = Pe^{xB}P^{-1}$として計算すればよい.

\begin{example}
    $A \coloneqq \begin{pmatrix} -1 & 2 \\ -4 & 5 \end{pmatrix}$とするとき,$e^{xA}$と$\bm{y}'=A\bm{y}$の一般解,そして初期値問題$\bm{y}(0)=\begin{pmatrix} 1 \\ 3 \end{pmatrix}$に対する解を求める.

    固有方程式$\phi(t)=\det (A-tI)=0$の解を求める.$\phi(t)=(t-1)(t-3)$であるから,$t=1,3$が解である.

    固有値1に対する固有ベクトルを求めよう.$\bm{y} \coloneqq (x\; y)^\top$とする.
    \[ \begin{pmatrix}
        -1-1 & 2 \\
        -4 & 5-1
    \end{pmatrix}
    \begin{pmatrix}
        x \\ y
    \end{pmatrix}
    =\bm{0}\]
    をとく.計算の末,固有ベクトルとして$(x\; y)^\top=(1 \; 1)^\top$がとれることがわかる.

    固有値3に対する固有ベクトルを求めよう.$\bm{y} \coloneqq (x\; y)^\top$とする.
    \[ \begin{pmatrix}
        -1-3 & 2 \\
        -4 & 5-3
    \end{pmatrix}
    \begin{pmatrix}
        x \\ y
    \end{pmatrix}
    =\bm{0}\]
    をとく.計算の末,固有ベクトルとして$(x\; y)^\top=(1 \; 2)^\top$がとれることがわかる.

    $\bm{u}_1 \coloneqq (1\; 1)^\top, \; \bm{u}_2 \coloneqq (1\; 2)^\top$として,$P^{-1} \coloneqq \begin{pmatrix} 1 & 1 \\ 1 & 2\end{pmatrix}$とおく.このとき,$P^{-1}=\begin{pmatrix} 2 & -1 \\ -1 & 1 \end{pmatrix}$となる.

    さて,$B \coloneqq P^{-1}AP = \begin{pmatrix} 1 & 0 \\ 0 & 3 \end{pmatrix}$とおく.ここで
    \[ e^{xB} = \begin{pmatrix} e^x & 0 \\ 0 & e^{3x} \end{pmatrix} \]
    であるから
    \begin{align*}
        e^{xA} &= Pe^{xB}P^{-1} \\
        &= 
        \begin{pmatrix}
            1 & 1 \\
            1 & 2
        \end{pmatrix}
        \begin{pmatrix}
            e^x & 0 \\
            0 & e^{3x}
        \end{pmatrix}
        \begin{pmatrix}
            2 & -1 \\
            -1 & 1
        \end{pmatrix} \\
        &= 
        \begin{pmatrix}
            2e^x-e^{3x} & -e^x+e^{3x} \\
            2e^x-2e^{3x} & -e^x+2e^{3x}
        \end{pmatrix}
    \end{align*}
    を得る.よって,一般解は,$\bm{c} \coloneqq (c_1 \; c_2)^\top$として
    \[ \bm{y}(x) = e^{xA}\bm{c}\]
    となる.初期値問題については
    \begin{align*}
        \bm{y}(x) &= 
        \begin{pmatrix}
            2e^x-e^{3x} & -e^x+e^{3x} \\
            2e^x-2e^{3x} & -e^x+2e^{3x}
        \end{pmatrix}
        \begin{pmatrix}
            1 \\ 3
        \end{pmatrix} \\
        &= -e^x\begin{pmatrix} 1 \\ 1 \end{pmatrix} + 2e^{3x} \begin{pmatrix} 1 \\ 2 \end{pmatrix} \\
        &= -e^x \bm{u}_1 + 3e^{3x} \bm{u}_2
    \end{align*}
    となる.
\end{example}

\begin{remark}
    $-e^x \bm{u}_1 + 3e^{3x} \bm{u}_2$をみればわかるように,解が$\bm{u}_1, \; \bm{u}_2$という行列$A$の固有ベクトルで表されている.
\end{remark}

\begin{homework*}
    $A \coloneqq \begin{pmatrix}
        2 & -1 \\ 
        -2 & 3
    \end{pmatrix}$だとどうなるか,各自計算せよ.
\end{homework*}

