\section{微分演算子(その2)・行列の指数関数(その1)}
\subsection{Maclaurineの方法}
次回述べなかったMaclaurineの方法を述べる.

$L(\lambda) = \lambda^p M(\lambda), \; M(\lambda) \neq 0$とする.$M(\lambda)$のMaclaurine展開を用いて$L(\lambda)^{-1}$が
\[ L(\lambda)^{-1} = \lambda^{-p} (b_0+b_1\lambda+b_2\lambda^2+\cdots)\]
とかけたとする.そして,$f(x)$が$k$次以下の整式でかけたとする.このとき
\[ L(D)^{-1}f = (b_0+b_1\lambda+b_2\lambda^2+\cdots+b_kD^k)f(x).\]

前回の例題において,$f(x)=xe^{2x}$であったが,これは明らかに整式ではない.したがって今回の方法が使えないように見える.さて,
\begin{align*}
    (D+1)^{-1} xe^{2x} &\beceq{\text{推移法則}} e^{2x} (D+1+2)^{-1} x \\
    &= e^{2x} (D+3)^{-1}x,
\end{align*}
という式に着目しよう.$xe^{2x}$を直接マクローリン展開するのではなく,「$(D+3)^{-1}$」を強引にマクローリン展開するのである.

今,$(\lambda+3)^{-1} = \frac{1}{3}\frac{1}{1+\lambda/3} = \frac{1}{3}(1-\frac{\lambda}{3}+(\frac{\lambda}{3})^2 + \cdots)$となるから
\begin{align*}
    (D+1)^{-1}xe^{2x} &= e^{2x}(D+3)^{-1}x \\
    &= e^{2x} \cdot \frac{1}{3}(1-\frac{D}{3}+\frac{D^2}{9}-\cdots)x \\
    &= e^{2x} \cdot \frac{1}{3}(x-\frac{1}{3}) \\
    &= \frac{1}{9}(3x-1)e^{2x}.
\end{align*}

次に,$(D-1)^2 \left(e^{2x}\cdot \frac{3x-1}{9}\right)$を計算する.
\begin{align*}
    (D-1)^2 \left(e^{2x}\cdot \frac{3x-1}{9}\right) \beceq{\text{推移法則}} \frac{1}{9}e^{2x}(D+1)^{-2}(3x-1).
\end{align*}
$(\lambda+ 1)^{-2}$を知るためには
\begin{align*}
    (\lambda+1)^{-2} &= \left(\frac{1}{\lambda+1}\right)^2 \\
    &= (1-\lambda+\lambda^2-\lambda^3+\cdots)^2
\end{align*}
を計算すればよい.また,$\frac{d}{d\lambda}(1+\lambda)^{-1} = -(1+\lambda)^{-2}$より
\begin{align*}
    (1+\lambda)^{-2} &= -\frac{d}{d\lambda}(1+\lambda)^{-1} \\
    &= -\frac{d}{d\lambda}(1-\lambda+\lambda^2-\lambda^3+\cdots) \\
    &= 1-2\lambda+ 3\lambda^2-4\lambda^3+\cdots.
\end{align*}
ちなみに,直接$(1-\lambda+\lambda^2-\lambda^3+\cdots)^2$を計算してもよい.よって
\begin{align*}
    (D+1)^{-2}(3x-1) &= (1-2D)(3x-1) \\
    &= 3x-7.
\end{align*}
となり,$y=\frac{1}{9}e^{2x}(3x-7)$を得る.

\begin{note*}[なんで$(1-2D)$で打ち切ったの?]
    $f(x)=3x-1$は高々1次であるから,$D^2$以降を作用させたとしても,その計算結果は0となる.したがって,1次までで十分である.
\end{note*}

別解答としては,次のようなものが挙げられる:
\begin{align*}
    (D+1)^{-2} (3x-1) &= (D+1)^{-1} (D+1)^{-1} (3x-1) \\
    &\beceq{\text{Maclaurine}} (D+1)^{-1} \left(\textcolor{red}{(1-D)}(3x-1)\right) \\
    &= (D+1)^{-1}(3x-4) \\
    &\beceq{\text{Maclaurine}} \textcolor{red}{(1-D)}(3x-4) \\
    &= 3x-7.
\end{align*}
いずれの方法にせよ,計算できるようにすることが大事である.テストで必ず出題される.


\subsection{定数係数の連立微分方程式について}
次のような連立微分方程式を考えたい:
\begin{equation} \label{eq:14-連立}
    \begin{cases}
        y_1' = 3y_1 + 4y_2 \\
        t'_2 = 2y_1 + y_2.
    \end{cases}
\end{equation}
$\bm{y} \coloneqq (y_1 \; y_2)^\top$とすれば,\eqref{eq:14-連立}は
\[ 
\begin{pmatrix}
    y_1 \\ y_2
\end{pmatrix}
= 
\begin{pmatrix}
    3 & 4 \\ 1 & 1
\end{pmatrix}
\begin{pmatrix}
    y_1 \\ y_2
\end{pmatrix}.\]
のように書ける.$A \coloneqq
\begin{pmatrix}
    3 & 4 \\ 1 & 1
\end{pmatrix},$とおけば,$\bm{y}' = A\bm{y}$となる.

復習をしよう.$y'=ay$の解は$y=ce~{ax}$であった.今回は$\bm{y}' = A\bm{y}$と形は似ているが,係数は行列だし,$y$はベクトルである.ここで,次のような疑問が浮かんでくるのは自然であろう:
\begin{question*}
    $e^{ax}$に相当するもの,つまり$\bm{e^{xA}}$なるものは存在するのか.
\end{question*}
ちなみに,$\bm{e^{xA}}$を\textbf{行列の指数関数}という.

次の微分方程式を考えよう:
\begin{equation}
    \bm{y}' = A\bm{y}. \label{eq:14-H}
\end{equation}
ただし,$\bm{y} \coloneqq (y_1(x) \; y_2(x) \; \cdots \; y_n(x))^\top$,$A=(a_{ij})$を定数行列とする.\eqref{eq:14-H}は次のようにも書ける:
\begin{equation}
    D\bm{y} = A\bm{y}.
\end{equation}
今,$\bm{y}=e^{\lambda x}\bm{c}$が解だと思って\eqref{eq:14-H}に代入してみよう.
\begin{align*}
    D\bm{y} &= \lambda e^{\lambda x} \bm{c}, \\
    A\bm{y} &= e^{\lambda x} \cdot A\bm{c}.
\end{align*}
したがって
\begin{align*}
    \lambda e^{\lambda x} \bm{c} &= A(e^{\lambda x} \bm{c}) \\
    (A-\lambda I)(e^{\lambda x}\bm{c}) &= \bm{0} \\
    e^{\lambda x}(A-\lambda)\bm{c} &= \bm{0}.
\end{align*}
ここで,$I$は単位行列である.
\begin{remark}
    \eqref{eq:14-H}の自明でない解を見つけるためには,$(A-\lambda)\bm{c} = \bm{0}$をみたす$\bm{c} \neq \bm{0}$を探せばよい.
\end{remark}
このとき,$\lambda$を固有値,$\lambda$に対応する$\bm{c}$を固有ベクトルという.$(A-\lambda)\bm{c} = \bm{0}$が$\bm{c} \neq \bm{0}$なる解をもつための必要十分条件は$\det (A-\lambda) = 0$である.このことがスッとこない場合は,線形代数を復習すること.
\begin{note*}
    $\det (A-\lambda I) \neq 0$だとすると,逆行列$(A-\lambda I)^{-1}$が存在するから
    \[ (A-\lambda I)^{-1} (A-\lambda I) \bm{c} = \bm{c} = \bm{0}\]
    となり,$\bm{c} = \bm{0}$という自明な解が得られてしまう.
\end{note*}

さて,\eqref{eq:14-H}の初期値問題
\[ \begin{cases}
    \bm{y}' = A\bm{y}, \\
    \bm{y}(0) = \bm{\alpha}
\end{cases}\]
を考えよう.ここで,指数関数のマクローリン展開について復習しておこう.
\begin{remind}[指数関数の]
    $e^{ax}$のMaclaurine展開は次である:
    \begin{align*}
        e^{ax} &= 1 + ax + \frac{(ax)}{2!} + \frac{(ax)^3}{3!} + \cdots \\
        &= \sum_{n=0}^{\infty} \frac{x^k}{k!} a^k.
    \end{align*}
\end{remind}
形式的に,次のようにおいてみる.つまり,$a$を行列$A$で置き換えてみる.
\begin{align*}
    e^{xA} &\coloneqq I + xA + \frac{1}{2!}(xA)^2 + \cdots + \frac{1}{n!}(xA)^n + \cdots \\ 
    &= \sum_{k=0}^{\infty} \frac{x^k}{k!}(A^k).
\end{align*}
これを\textbf{行列の指数関数}と定義し,$e^{xA}$とかく.
\begin{note*}
    右辺の級数は,"良い"収束をする.要は,一様収束のこと.ここらへんについては\ref{book:takahashi}などに書いてある.
\end{note*}
$A^k$が$n$次正方行列であるから,右辺の級数も同様である.つまり$e^{xA}$は$n$次正方行列である.

ここで,$e^{xA}$の性質について述べよう.
\begin{proposition}
    行列の指数関数$e^{xA}$は次のような性質をもつ:
    \begin{enumerate}[label=(\arabic*)]
        \item $e^{x O} = I$.ここで$O$は零行列である.
        \item $(e^{xA})^{-1} = e^{-xA}.$
        \item $e^{(x+y)A} = e^{xA} \cdot e^{xA} = e^{yA} \cdot e^{xA}.$
        \item $e^{x(aI+A)} = e^{ax} e^{xA}.$
        \item $\displaystyle \frac{d}{dx}(e^{xA}) = A e^{xA}.$
    \end{enumerate}
\end{proposition}
今,$e^{xA}$に初期値$\bm{\alpha}$を作用させた$\bm{y}(x)=e^{xA}\bm{\alpha}$について
\begin{align*}
    \bm{y}(0) &= e^{0A} \cdot \bm{\alpha} = e^0 \pmb{\alpha} = \bm{\alpha}, \\
    \bm{y}(x) &= \frac{d}{dx}(e^{xA}\bm{\alpha}) = Ae^{xA}\bm{\alpha} = A\bm{y}(x),
\end{align*}
つまり$\bm{y}(0)=\bm{\alpha}, \; \bm{y}(x)=A\bm{y}(x)$であるから,$\bm{y}(x)=e^{xA}\bm{\alpha}$が初期値問題の解になっている.

\begin{theorem}
    次の初期値問題の解は,$\bm{y}(x)=e^{xA}\alpha$となる:
    \[ \begin{cases}
        \bm{y}' = A\bm{y}, \\
        \bm{y}(0) = \bm{\alpha}.
    \end{cases}\]
    存在定理から,一意性も保証されている.
\end{theorem}
さらに,非同次問題についても考えよう.
\begin{equation} \label{eq:14-I}
    \begin{cases}
        \bm{y}(x) = A\bm{y} + \bm{b}, \\
        \bm{y}(0) = \bm{\alpha}.
    \end{cases}
\end{equation}

\begin{theorem}
    \eqref{eq:14-I}の解は,次となる:
    \[ \bm{y}(x) = e^{xA}\bm{\alpha} + \int_{0}^{x} e^{(x-t)A}\bm{b}(t) \; dt.\]
    第1項目は基本解部分(余因子),第2項目は特殊解となる.
\end{theorem}

\vfill
次回予告をしておこう.
\begin{question*}
    行列$A$に対する,$e^{xA}$は何者なのか.そしてそれはいかにして求めるのか.
\end{question*}