\documentclass[dvipdfmx,a4,11pt]{jsarticle}
\pagestyle{headings}

\usepackage{amsmath,amssymb,amsthm,amsfonts}
\usepackage{mathtools}
\usepackage{mathrsfs}
\usepackage{bm}
\usepackage{braket}

\usepackage{framed}
\setlength{\topsep}{1pt} %framed環境の空白調整

\usepackage{enumitem}
\usepackage{comment}

\usepackage[bookmarks=true,bookmarksnumbered,hidelinks,colorlinks]{hyperref}
\usepackage{pxjahyper}

%sectionを明朝体にする
\usepackage{titlesec}
\titleformat*{\section}{\normalfont\Large\bfseries} % 明朝体・太字
\titleformat*{\subsection}{\normalfont\large\bfseries}
\titleformat*{\subsubsection}{\normalfont\bfseries}

\usepackage{tocloft}
%目次のセクションフォントを明朝体(normalfont)に変更
%\renewcommand{\cftsecfont}{\normalfont}
%\renewcommand{\cftsubsecfont}{\normalfont}
%\renewcommand{\cftsubsubsecfont}{\bfseries\normalfont}

% 目次内の太字を解除(太字が嫌な場合)
%\renewcommand{\cftsecpagefont}{\normalfont}
%\renewcommand{\cftsubsecpagefont}{\normalfont}



%%%------定理環境------%%%
\theoremstyle{definition}
\newtheorem{mydefinition}{定義}[section]
\newtheorem{mytheorem}[mydefinition]{定理}
\newtheorem{myproposition}[mydefinition]{命題}
\newtheorem{mylemma}[mydefinition]{補題}
\newtheorem{mycornally}[mydefinition]{系}
\newtheorem{myexample}[mydefinition]{例}
\newtheorem{attention}{注意}
\newtheorem{remind}{Remind}[section]
\newtheorem{remark}{Remark}[section]
\newtheorem*{myclaim*}{Claim}
\newtheorem*{myquestion*}{Question}
\newtheorem*{mynote*}{Note}
\newtheorem*{mytalk*}{Talk}

\newtheorem*{homework*}{homework}

\renewcommand{\qedsymbol}{$\blacksquare$}
\renewcommand{\proofname}{\textbf{【証明】}}

\newenvironment{definition}{\begin{mydefinition}}{\hfill $\square$ \end{mydefinition}}
\newenvironment{theorem}{\begin{mytheorem}}{\hfill $\square$ \end{mytheorem}}
\newenvironment{proposition}{\begin{myproposition}}{\hfill $\square$ \end{myproposition}}
\newenvironment{lemma}{\begin{mylemma}}{\hfill $\square$ \end{mylemma}}
\newenvironment{cornally}{\begin{mycornally}}{\hfill $\square$ \end{mycornally}}
\newenvironment{example}{\begin{myexample}}{\hfill $\square$ \end{myexample}}
\newenvironment{claim*}{\begin{myclaim*}}{\hfill $\square$ \end{myclaim*}}
\newenvironment{question*}{\begin{myquestion*}}{\hfill $\square$ \end{myquestion*}}
\newenvironment{note*}{\begin{leftbar}\begin{mynote*}}{\end{mynote*}\end{leftbar}}
\newenvironment{talk*}{\begin{leftbar}\begin{mytalk*}}{\end{mytalk*}\end{leftbar}}

\numberwithin{equation}{section}

%%%------コマンド(数式)------%%%
\newcommand{\st}{\text{s.t.}} %such that
\newcommand{\ie}{\text{i.e.}}

\newcommand{\defiff}{\overset{\mathrm{def}}{\iff}} %定義(同値記号)
\newcommand{\defeq}{\overset{\mathrm{def}}{=}} %定義(イコール)
\newcommand{\beceq}[1]{\overset{#1}{=}}
\newcommand{\inprod}[2]{\Braket{#1,#2}} %内積<a,b>
\newcommand{\aew}[1]{#1\text{-a.e.}}


%%%------コマンド(その他)------%%%
\newcommand{\checking}{\noindent \fbox{確認}}


%%%------演算子------%%%
\DeclareMathOperator{\tr}{\mathrm{tr}} %trace
\DeclareMathOperator{\grad}{\mathrm{grad}} %gradien
\DeclareMathOperator{\diver}{\mathrm{div}} %divergence
\DeclareMathOperator{\re}{\mathrm{Re}} %Real part
\DeclareMathOperator{\im}{\mathrm{Im}} %Imaginary part
\DeclareMathOperator{\rank}{\mathrm{rank}} %rank



%%%------タイトル------%%%
\title{{\Large 2025年 前期} \\ 解析学2(常微分方程式論)}
\author{講義担当者:中川和重 \\ 
講義ノート作成者:斎藤尚}
\date{最終更新日:\today}


\setcounter{section}{-1} %section0からスタート
\begin{document}
\maketitle
\tableofcontents

\clearpage

%%%------本題------%%%
\section{はじめに}
\subsection{参考書}
中川先生は「図書館で良い本に出会ってね(にっこり)」といっていて,確かにそれはその通りなのだが,何を選べばいいのかよくわからない学生もいるはずだ.そういう人のために,個人的これはよいだろうというものを以下に挙げる:
\begin{enumerate}[label=\lbrack\arabic*\rbrack]
    \item 高橋陽一郎,力学と微分方程式,岩波書店.\label{book:takahashi}
    \item 柳田英二・栄伸一郎,常微分方程式論,朝倉書店.\label{book:yanagida}
    \item 工学系の何か.
\end{enumerate}
先生に確認はしていないが,おそらく授業のおおかたは(2)を参考にしていると思われる.そう主張するのは,第15回で扱った例題が,その本に丸々載っていたからである.

\subsection{記法}
本講義ノートでは以下のような記号を用いる.
\begin{itemize}
    \item $\mathbb{N},\; \mathbb{Z},\; \mathbb{Q},\; \mathbb{R},\; \mathbb{C}$:左から順に,自然数,整数,有理数,実数,複素数を表す.
    \item Note:定義や定理への補足をNoteに記す.
    \item Talk:数学に関連した小話または本当に関係のない話はここに記す.
    \item $\st$:such thatの略である.
    \item $\ie$:id estの略である.
\end{itemize}


\clearpage
\section{ガイダンス} %第1回
すぐ終わった.
\begin{talk*}[中川先生の授業って...]
    中川先生の担当する授業の初回は,大体10分とか15分で終わる.初回からぶっぱなせばいいのにと,複素関数論とこの授業を履修していて思うのだが,そうしないのはなにかこだわりがあるのか,はたまた面倒なのかどっちなのか.

    以前,大学院生のTAと話していたのを盗み聞きしたことがある.そのとき「最初の方でゆったりやるから後半つめつめなんだよねー」的なことをぼやいていた.初回からぶっぱなす先生をわたしはみたい.
\end{talk*}


\clearpage
%第2回
\section{放射性崩壊と数理モデル} %第2回
\subsection{1階微分方程式}
放射性崩壊の法則は「放射性物質の崩壊速度は,そのとき存在する物質量に比例する」ということを主張する.これは微分方程式で表せば,以下のように記述される:
\begin{equation}
    \frac{dN}{dt} = -\lambda N.
\end{equation}

$N(t)$を時刻$t$における原子数,$N(t+\Delta t)$を時刻$t+\Delta t$における原子数とする.このとき,$N(t)-N(t+\Delta t)$は$\Delta t$の時間で減少した原子数を意味する.一方で,法則より,この数は$\lambda N(t)\Delta t$とかける.したがって以下を得る:
\begin{align*}
    N(t)-N(t+\Delta t) &= \lambda N(t)\Delta t \\
    \frac{N(t+\Delta t)-N(t)}{\Delta t} &= -\lambda N(t).
\end{align*}
$\Delta t \to 0$とすると,$Dn/dt =-\lambda N$となる(cf: E.\;Ruthorford et al. (1903) \footnote{Ruthorfordは「ラザフォード」とよむ,})

ここで質問をしよう.$dN/dt = -\lambda N$はどう解けばよいのだろうか.答えは以下の通り:
\begin{proof}[\textbf{答え}]
    $\frac{1}{N} \; dN = -\lambda \; dt$であるから,両辺を積分して$\log|N| = -\lambda t + C$.したがって$N = Ae^{-\lambda t}$となる.ここで$A \coloneqq e^{\pm C}$とおいた.
\end{proof}


\clearpage
\section{オンライン課題} %第3回
人工モデルに関して,エクセルで実験.


\clearpage
%第4回
\section{色々な常微分方程式,逐次近似法} %第4回
\subsection{常微分方程式}
次のような常微分方程式を考えよう:
\begin{example}[1階の常微分方程式]
    $y'(x) - y(x) = 0$において,$y=e^x$は解である.
\end{example}

\begin{example}[2階の常微分方程式] \label{ex:2order-ode}
    $y''(x) - y(x) = 0$において,$y=\sin x, \; y=\cos x$は解である.
\end{example}

\begin{example}[$n$階の常微分方程式]
    $f(x)$を0を含む区間で定義された関数とする.$y^{(n)} = f(x)$については,$n$回積分すればもとまりそうである.では実際に積分してみよう.
    \begin{align*}
        \int_{0}^{x} y^{(n)}(x) \; dx &= \int_{0}^{x} f(x) \; dx \\
        y^{(n-1)} &= \int_{0}^{x} f(x) \; dx + c_1 \quad (c_1 \coloneqq f^{(n-1)}(0)).
    \end{align*}
    同様にして
    \begin{align*}
        y^{(n-2)}(x) &= \int_{0}^{x} \left( \int_{0}^{x_{n-1}} f(x_1) \; dx + c_1 \right) \; dx + c_2 \\
        &= \int_{0}^{x} \int_{0}^{x_{n-1}} f(x_n) \; dx_n \; dx_{n-1} + c_1x + c_2.
    \end{align*}
    これをずっと続けると,次のような結果を得る:
    \begin{equation}
        y(x) = \int_{0}^{x} \int_{0}^{x_1} \cdots \int_{0}^{x_{n-1}} f(x_n) \; dx_n \; dx_{n-1} \cdots dx_1 + \sum_{k=1}^{n} \frac{c_k}{(n-k)!}x^{n-k}.
    \end{equation}
    ただし,$c_1,\ldots,c_n$は任意定数である.

    
\end{example}

一般に,次のことが言える:
\begin{itemize}
    \item $n$階の常微分方程式は$n$つの任意定数を含む $\longrightarrow$ 一般解,
    \item 定数に特別な値を代入して得られる解 $\longrightarrow$ 特殊解,
    \item どんな値を代入しても得られない解 $\longrightarrow$ 特異解.
\end{itemize}

\begin{example}
    \ref{ex:2order-ode}において,$y=c_1\cos x + c_2 \sin x$は$y'' + y= 0$をみたす.これは\textbf{一般解}.ここで,$y_1=\cos x + 2\sin x$について,${y_1}''+y_1 = 0$をみたすから,$y_1$は解である.加えて,$y_1$は$(c_1,c_2)=(1,2)$としたものだから,これは特殊解である.

    $\tilde{y} \coloneqq c_1(\cos x + \sin x) + c_2 \sin x$も,任意の$c_1,c_2$について$y''+y=0$をみたすから,これは一般解である.このとき,$y_1=\cos x + 2\sin x$は$(c_1,c_2)=(1,1)$としたときの特殊解である.
\end{example}

\begin{remark}
    一般解のによって,$c_1,c_2$の組は変わる.
\end{remark}

\begin{homework*}
    $y \coloneqq c_1 e^{ix} + c_2 e^{-ix}$は$y''+y=0$の一般解となることを示せ.
\end{homework*}

\begin{example}
    $y'=xy + xy^2$を考える.このとき,$y_1=\frac{ce^{x^2/2}}{1-ce^{x^2/2}}, \; y_2=\frac{e^{x^2/2}}{c-e^{x^2/2}}$は解である.

    実際,これらを代入してなりたつことを示せばよい.だが,これはガリガリ計算したくない人にとっては苦痛である.そこで,次に見せるような方法だと,それはある程度解消されるかもしれない.

    まず,次がなりたつことは大丈夫だろう:
    \begin{align*}
        y_2 &= \frac{e^{x^2/2}}{c-e^{x^2/2}} \\
        (c-e^{x^2/2})y_2 &= e^{x^2/2}. \\
    \end{align*}

    次に,この両辺を微分すると,以下のような結果を得る:
    \begin{align*}
        -xe^{x^2/2}y_2 + (c-e^{x^2/2}){y_2}' &= xe^{x^2/2} \\
        (c-e^{x^2/2}){y_2}' &= xe^{x^2/2}(1+y_2) \\
        {y_2}' &= \frac{e^{x^2/2}}{c-e^{x^2/2}}(x+xy_2) \\
        {y_2}' &= xy_2 + x{y_2}^2.
    \end{align*}

    さて,$y \equiv 0, \; 1$は解となるのだが,一般解$y_1, \; y_2$とどのように関係しているかをみよう.

    $y_1$について
    \begin{itemize}
        \item $y \equiv 0$は$c=0$のときの特殊解である.
        \item $y \equiv -1$は特異解である.
    \end{itemize}
    $y_2$について
    \begin{itemize}
        \item $y \equiv 0$は特異解である.
        \item $y \equiv -1$は$c=0$のときの特殊解である.
    \end{itemize}
\end{example}

\begin{remark}
    特殊解or特異解は,対応する一般解とセットで考える.
\end{remark}

\subsection{逐次近似法}
\begin{example}
    $y'(x)-y(x)=0$を\textbf{逐次近似法}でとく(これをPicardの方法ともいう).ただし,$x=0$のとき$y(0)=c$とする.
    この常微分方程式を次のように記そう:
    \begin{equation}
        \begin{cases}
            y'(x)-y(x)=0, \\
            y(0)=c.
        \end{cases}
    \end{equation}

    これに対応する積分方程式は以下になる:
    \begin{align*}
        y(x)-y(0) &= \int_0^x y(x) \; dx \\
        y(x) &= \int_0^x y(x) \; dx + c.
    \end{align*}
    
    さて,関数列$\{y_k(x)\}$を次で定める.これは漸化式である.
    \begin{equation}
        \begin{cases}
            y_0(x) = c, \\
            y_k(x) = \int_{0}^{x} y_{k-1}(x) \; dx + c.
        \end{cases}
    \end{equation}

    具体的に計算してみよう.
    \begin{align*}
        y_1(x) &= c + \int_{0}^{x} y_0(x) \; dx = c(1+x), \\
        y_2(x) &= c + \int_{0}^{x} y_1(x) \; dx = c(1+x+\frac{1}{2}x^2), \\
        y_3(x) &= c + \int_{0}^{x} y_2(x) \; dx = c(1+x+\frac{1}{2}x^2+\frac{1}{3!}x^3).
    \end{align*}

    よって
    \begin{equation}
        y_k(x) = c(1+x+\frac{1}{2}x^2+\cdots+\frac{1}{k!}x^k) = \sum_{j=0}^{k} \frac{x^j}{j!}
    \end{equation}
    と推測できる.
\end{example}

\begin{note*}[この推測は正しいの?]
    この推測が正しいことは,帰納法を用いて示せる.実際,$k$で正しいとして
    \begin{align*}
        y_{k+1}(x) &= c + \int_{0}^{x} c\sum_{j=0}^{k} \frac{x^j}{j!} \; dx \\
        &= c + c \sum_{j=o}^{k} \frac{x^{j+1}}{(j+1)!} \\
        &= c \sum_{j=0}^{k} \frac{x^j}{j!}.
    \end{align*}
\end{note*}

$k \to \infty$とすれば,次を得る:
\begin{equation}
    y_{\infty}(x) = c \sum_{j=0}^{\infty} \frac{x^j}{j!} = ce^x.
\end{equation}

このようにして,$y=ce^x$を構成できる.\footnote{この方法を\ref{book:takahashi}で見たとき感動したんですけど,皆さんも感動しませんでしたか?}

\begin{example}
    $y=c_1\cos x + c_2\sin x$は次の微分方程式を満たす:
    \begin{equation}
    \begin{cases}
        y''(x)+y(x) = 0, \\
        y(x) = c_1, \\
        y'(x) = c_2.
    \end{cases}
    \end{equation}

    $y'(x)=z(x)$とおく.$z'=y''=-y$であり,$z(0)=y'(0)=c_2$である.これらを踏まえて,問題の常微分方程式は
    \begin{equation}
        \begin{cases}
            y'(x)= z(x), \\
            y(0) = c_1, \\
            z'(x) = -y, \\
            z(0) = c_2.
        \end{cases}
    \end{equation}
    という,$y$と$z$の1階の常微分方程式系で書ける.

    これに対応する積分方程式は
    \begin{equation}
        \begin{cases}
            y(x) = c_1 + \int_{0}^{x} z(x) \; dx, \\
            z(x) = c_2 - \int_{0}^{x} y(x) \; dx
        \end{cases}
    \end{equation}
    となる.関数列$\{y_k\}, \{z_k\}$を以下で定める:
    \begin{equation}
        \begin{cases}
            y_k = c_1 + \int_{0}^{x} z_{k-1} \; dx, \quad y_0 = c_1, \\
            z_k = c_2 - \int_{0}^{x} y_{k-1} \; dx, \quad z_0 = c_2.
        \end{cases}
    \end{equation}
    試しに$k=1$だけ計算しよう.すると,
    \begin{align*}
        y_1 &= c_1 + \int_{0}^{x} z_0 \; dx = c_1 + c_2 x, \\
        z_1 &= c_2 - \int_{0}^{x} y_0 \; dx = c_2 - c_1 x
    \end{align*}
    となる.これを繰り返すと,以下のような結果を得る:
    \begin{equation}
        y_k(x) = c_1 \sum_{j=0}^{[\frac{k}{2}]} \frac{(-1)^j}{(2j)!} x^j + c_2 \sum_{j=0}^{[\frac{k-1}{2}]} \frac{(-1)^j}{(2j+1)!} x^j.
    \end{equation}
    $k \to \infty$とすると
    \begin{align}
        y_{\infty} &= c_1 \sum_{j=0}^{\infty} \frac{(-1)^j}{(2j)!} x^j + c_2 \sum_{j=0}^{\infty} \\
        &= c_1 \cos x + c_2 \sin x.
    \end{align}
\end{example}

\begin{note*}[ガウス記号]
    $\sum$の上に$[\frac{k}{2}], \; [\frac{k-1}{2}]$という記号が出てきている.これは\textbf{ガウス記号}と呼ばれるものである.$[\frac{k}{2}]$の値は,\textbf{$\frac{k}{2}$を超えない最大の整数}である.

    つまり,$[a]$はある整数$n$で$n \leq a <n+1$となるもののことである.
    
    ガウス記号の例をみよう.$[3.4]$であれば,$[3.4]=3$となる.$[-3.4]$であれば$[-3.4]=4$となる.$[-3.4]=3$だと,$-3$は$-3.4$を超える整数であるため,不適である.
\end{note*}
さて,これが3階,4階と階数が増えたらどうなるだろうか.それについては,各自でぼんやりと考えればよい(とわたしは思う).

とにかく,\textbf{構成できるという事実が大切}ということを,実感してほしい.


\clearpage
%第5回
\section{逐次近似法(続き)・求積法} %第5回
\subsection{逐次近似法(続き)}
\begin{definition}[正規形]
    $n$階のODEが\textbf{正規形}であるとは,$y^{(n)}$についてとける,すなわち
    \begin{equation}
        y^{(n)} = f(x, y, y', \ldots, y^{(n-1)}) \label{eq:ast}
    \end{equation}
    とできるときをいう.
\end{definition}

\begin{example}
    $y'=ay, \; y''+y=0$は正規形である.
\end{example}

\eqref{eq:ast}について,$y(x) \coloneqq y_1(x), \; y'(x) \coloneqq y_2(x), \ldots, y^{(n-1)}(x) \coloneqq y_n(x)$とおくとき,両辺をそれぞれ微分すると
\[ y'(x) = {y_1(x)}' = y_2, \; y''(x) = {y_2(x)}' = y_3, \ldots, y^{(n)}(x) = y_n'.\]
となり,これをベクトルを用いて表現すると以下のようになる:
\begin{equation}
    \frac{d}{dx}
    \begin{pmatrix}
        y_1(x) \\ y_2(x) \\ \vdots \\ y_{n-1}(x) \\ y_n(x)
    \end{pmatrix}
    =
    \begin{pmatrix}
        y_2(x) \\ y_3(x) \\ \vdots \\ y_n(x) \\ f(x,y,\ldots,y_n)
    \end{pmatrix}.
\end{equation}

$\bm{y}(x) \coloneqq (y_1 \; \cdots \; y_n)^\top$とおくと
\begin{equation}
    \bm{y}'(x) = \bm{f}(\bm{x},y) \label{eq:dagger}
\end{equation}
となる(ベクトルを用いているが,正規形であることに注意しよう).

\eqref{eq:dagger}を積分すると
\begin{align*}
    \bm{y}(x) &= \int \bm{f}(t,y(t)) \; dt + \bm{c}, \\
    \bm{y}(x) &= \int \bm{f}(t,z(t)) \; dt + \bm{c}
\end{align*}
となる.ここで,$\bm{c}$は定ベクトルである(全ての成分が定数であるようなベクトルのこと).こkで,次のことが成り立つ:
\begin{center}
    $\bm{y}$が\eqref{dagger}の解である$\iff$$\bm{y}=\bm{z}$.
\end{center}
この考え方を利用したのが,\textbf{ピカールの逐次近似法}である.

$x=a$のとき$\bm{y}=\bm{c}$となる解は次のようになる:
\[ y(x) = \bm{c} + \int_{a}^{x} \bm{f}(t,\bm{y}(t)) \; dt.\]
関数列$\{y_n\}$を次のように帰納的に定める:
\begin{align*}
    y_0(x) &= \bm{c}, \\
    y_k(x) &= \bm{c} + \int_{a}^{x} \bm{f}(t,y_{k-1}) \; dt.
\end{align*}
$\lim_{k \to \infty} y_k(x)$が存在すれば,それが解となる.ただし,$\bm{f}$には条件が必要である.
\begin{homework*}
    $\bm{f}$について必要となる条件について考察せよ.
\end{homework*}

\subsection{求積法}
\subsubsection{変数分離形}
\begin{equation}
    y'(x) = P(x) Q(y).  \label{eq:ast-0}
\end{equation}
ここで,$P(x)$は$x$についての式,$Q(y)$は$y$についての式である.

変数分離形の解き方については以下のように手順がある:
\begin{enumerate}[label=(\arabic*)]
    \item $Q(y_0)=0$となる解$y_0$はあるのか.もしあるなら,$y(x)=y_0$は解となる($\because$ $y'=0$から$y=y_0$が従う).
    \item $y(x) \neq y_0$のとき(すなわち$Q(y) \neq 0$)
    \begin{equation}
        \frac{1}{Q(y)} dy = P(x) \label{eq:ast-1}
    \end{equation}
    となる.$G(y) \coloneqq \int \frac{1}{Q(y)} \; dy$とおく.このとき,$y$は$x$の関数より,次が従う:
    \[ \frac{d}{dx} G(y) = \frac{dG}{dy} \frac{dy}{dx} = \frac{1}{Q(y)} y''(x).\]
    これを\eqref{eq:ast-1}に代入すると
    \[ \frac{d}{dx} G(y) = \int P(x) \; dx + C\]
    となる.両辺$x$で積分すると
    \[ G(y) = \int P(x) \; dx + C\]
    となる.したがって,我々は以下を得る:
    \[ \int \frac{1}{Q(y)} \; dy = \int P(x) \; dx + C.\]
    計算して,$y$について整理したものが解である.
\end{enumerate}

\begin{example}
    $y'=y$について考える.これは
    \begin{equation}
        \begin{cases}
            P(x) = 1, \\
            Q(y) = y
        \end{cases}
    \end{equation}
    の変数分離形である.先ほど述べた手順に沿って解を求めよう.

    (1) $Q(y)=0$となる解があるかどうかみよう.実際,$y_0=0$は解であることがわかる.

    (2) $Q(y) \neq 0$のとき$\frac{1}{y} \; dy = 1 \; dx$であるから
    \begin{align*}
        \log |x| &= x+c \\
        \iff |y| &= e^{x+c} \\
        \iff y &= \pm e^c e^x.
    \end{align*}
    $C \coloneqq \pm e^c$とおくと,$c \neq 0$で$y=Ce^x$となる.

    (1),(2)より,$y(x)=0$は$C=0$としたときに相当することがわかる.よって解は$y(x)=Ce^x$である.
\end{example}

\begin{example}
    $y'(x)=y^{2/3}$を考える.これは
    \begin{equation}
        \begin{cases}
            P(x) = 1, \\
            Q(y) = y^{\frac{2}{3}}
        \end{cases}
    \end{equation}
    の変数分離形である.こちらも,先ほど述べた手順に沿って解を求めよう.

    (1) $y_0=0$とすると,$Q(y_0)=0$よりこれは解である.

    (2) $Q(y) \neq 0 \quad \ie \quad y \neq 0$のとき
    \begin{align*}
        \int \frac{1}{3y^{2/3}} &= \int 1 \\
        y^{1/3} &= x+c \\
        y &= (x+c)^3.
    \end{align*}

    $y=0$はどの$c$にも対応しない.したがって$y=0, \; (x+c)^3$が解である.
\end{example}

\begin{example}
    $y' = xy + xy^2$を考える.これは
    \begin{equation}
        \begin{cases}
            P(x) = x, \\
            Q(y) = y+y^2
        \end{cases}
    \end{equation}
    の変数分離形である.

    (1) $y = 0, \; -1 $とすると,$Q(y_0)=0$よりこれらは解である.

    (2) $y(x) \neq y_0$のとき
    \begin{align*}
        \int \frac{1}{y+y^2} \; dy &= \int x \; dx \\
        \int \left( \frac{1}{y} - \frac{1}{y+1}\right) \; dx &= \int x \; dx \\
        \log |y| - \log |y+1| &= \frac{1}{2}x^2 + c \\
        \log \left|\frac{y}{y+1}\right| &= \frac{1}{2}x^2 + c \\
        \frac{y}{1+y} &= Ce^{x^2/2},
    \end{align*}
    となる.ただし$c \neq 0$とする.したがって$y=\frac{Ce^{x^2/2}}{1-Ce^{x^2/2}}$を得る.

    $c=0$が$y=0$に対応することがわかる.一方,$y=-1$に対応する$c$は存在しない.以上より,解は以下になる:
    \[ y = \frac{Ce^{x^2/2}}{1-Ce^{x^2/2}}, \; -1.\]
\end{example}

\subsubsection{同次形}
\begin{equation}
    y'(x) = f\left(\frac{y}{x}\right). \label{eq:同次形}
\end{equation}
同次形は,変数変換して変数分離形に帰着することができる.実際,$z \coloneqq \frac{y(x)}{x}$とすると$y(x) = x \cdot z(x)$となり,これらを両辺微分すると
\begin{align*}
    y'(x) = z(x) + xz'(z) &= f(z(x))  \\
    z'(x) &= \frac{f(z) - z}{x}.
\end{align*}
これは$\begin{cases} P(x)=x^{-1}, \\ Q(z)=f(z)-z \end{cases}$の変数分離形である.あとは,さきほど紹介した変数分離形を解く手順に従えば解を得ることができる.

\begin{example}
    $y'(x)=\frac{xy}{x^2+y^2}$を考える.
    \begin{align*}
        \frac{xy}{x^2+y^2} = \frac{y/x}{1+(y/x)^2}
    \end{align*}
    より,これは同次形である($\because z = y/x, \; f(z) \coloneqq \frac{z}{1+z^2}.$)
\end{example}


\clearpage
%第6回
\section{同次形とその具体例} %第6回
\subsection{同次形}
\begin{example}
    $y'=\frac{xy}{x^2+y^2}$を解く.

    $\frac{xy}{x^2+y^2}=\frac{y/x}{1+(y/x)^2}$より,これは同次形である.$z(x) \coloneqq y/x$とおく.$y=zx$より$y'=z+xz'$であるから
    \begin{align*}
        z+xz' &= \frac{z}{1+z^2} \\
        \ie \quad xz &= \frac{-z^3}{1+z^2}.
    \end{align*}
    したがって以下を得る:
    \[ z' = -\frac{1}{x} \cdot \frac{z^3}{1+z^2}.\]
    これは$P(x)=-1/x, \; Q(z)=\frac{z^3}{1+z^2}$変数分離形であるから,前回学習した方法で解を求めることができる.

    (1) $Q(z)=0$のとき,$z=0$である(これを\textbf{自明解}という).

    (2) $Q(z) \neq 0$のとき,両辺を$Q(z)$で割ると
    \begin{align*}
        \int \frac{1+z^2}{z^3} \; dz &= \int -\frac{1}{x} \; dx \textcolor{blue}{+c} \\
        \int \left(-\frac{1}{z^3} + \frac{1}{z}\right) \; dz &= -\log |x| + c \\
        \ie \quad -\frac{1}{2z^3} + \frac{1}{z} &= -\log |x| + c.
    \end{align*}

    (3) $y$と$x$の式に直そう.
    \begin{align*}
        -\frac{1}{2}\left(\frac{x}{y}\right)^2 + \log \left|\frac{y}{x}\right| &= -\log |x| + c \\
        -\frac{1}{2}\left(\frac{x}{y}\right)^2 + \log |y| - \log |x| &= -\log |x| + c \\
        \ie \quad -\frac{1}{2}\left(\frac{x}{y}\right)^2 + \log |y| &= c.
    \end{align*}

    よって,求める解は$y=0, \; -\frac{1}{2}\left(\frac{x}{y}\right)^2+\log |y| =c$である.
\end{example}

$P,Q$が多項式であり,$f\left(y/x\right) = Q(x,y)/P(x,y)$とかけているとき,次のことがなりたつ:
\begin{claim*}
    同次形であるためには,$P,Q$のすべての項が$x,y$について同じ次数$k$であればよい.
\end{claim*}

\begin{example}[$k=2$]
    $k=2$として,$P(x,y)=ax^2+bxy+cy^2, \; Q(x,y)=\alpha x^2 + \beta xy + \gamma y^2$のとき,分子分母を$x^k$で割ると
    \[ \frac{Q(x,y)}{P(x,y)} = \frac{\alpha + \beta (\frac{y}{x}) + \gamma (\frac{y}{x})^2}{a + b (\frac{y}{x}) + c (\frac{y}{x})^2}\]
    となる.
\end{example}

\begin{example} \label{example:homework}
    $y'=\frac{x-2y}{2x+y}$を解く.

    分子分母を$x$で割ると
    \[ y' = \frac{1-2\frac{y}{x}}{2+\frac{y}{x}} = \frac{1-2z}{2z+1}.\]
    となる.ここで$z \coloneqq y/x$とおいた.$y=xz$について両辺微分すると$y'=z+xz'$となるから
    \begin{align*}
        z+xz' &= \frac{1-2z}{2z+1} \\
        \ie \quad z' &= \frac{1}{x} \cdot \frac{-1-3z}{2+z}.
    \end{align*}

    さて,これは$P(x)=1/x, \; Q(z) = (-1-3z)/(2+z)$の変数分離形である.これを解くのは各自の宿題としよう\footnote{実際には講義中で演習として計算したが,この講義ノートは\LaTeX でつくられている故打ち込むのが面倒だから,宿題としている.}
\end{example}

\begin{homework*}
    例\ref{example:homework}を解け.
    
    ヒント:途中で$\int \frac{z+2}{z^2+4z-1} \; dz$の積分が出てきてこれに戸惑うかもしれない.これは次のように変形するとうまくいく:
    \[ \int \frac{z+2}{z^2+4z-1} \; dz = \int \frac{z+2}{(z+2)^2-5} \; dz = \frac{1}{2}\log |(z+2)^2-5|. \]

    解は$y/x=-2\pm \sqrt{5}, \; \frac{1}{2}\log|(y/x)^2+4(y/x)-1|=- \log|x| + c$となる.

    もっと簡潔に表したい場合は頑張って計算すること.最終的には$y^2+4xy-x^2=\tilde{c}$に落ち着く.
\end{homework*}

\subsection{同次形の方程式が必要か?(具体例は?)}
図があって書くのが大変なので省略.ここよりも同次形の解き方をマスターすることが大事.

\subsection{同次形に帰着できる問題}
先ほどの例でやったように,常微分方程式$y'=\frac{x-2y}{2x+y}$は同次形である.さて,これが$y'=\frac{x-2y+1}{2x+y+b}$といったように,分子分母に定数項が加わるとどうなるだろうか.これは同次形ではなくなる.ではどのようにして解けばよいか.それはある変数変換を行うことにより,解くことが可能になる.

$x-2y+1=0, \; 2x+y+b=0$をみたす$(x,y)$を$(\alpha,\beta)$と書くことにする.ここで,
\[ X \coloneqq x-\alpha, \; Y \coloneqq y-\beta\]
という変数変換を施す.すると,以下の計算結果が得られる:
\begin{align*}
    x-2y+a &= X-2Y, \\
    2x+y+b &= 2Y+X.
\end{align*}
$dy/dx = dY/dX$であるから
\[ \frac{dY}{dX} = \frac{X-2Y}{2Y+X}\]
を得る.\textbf{これは同次形である!}

ちなみに,$x-2y+1=0, \; 2x+y+b=0$をみたす$(x,y)$を求めるには,以下をとけばよい.これがピンとこない場合は,線形代数について復習すること.
\[ \begin{pmatrix}
    1 & -2 \\ 2 & 1
\end{pmatrix}
\begin{pmatrix}
    x \\ y
\end{pmatrix}
= \begin{pmatrix}
    a \\ b
\end{pmatrix}.\]

\begin{example}
    $\frac{dy}{dx} = \frac{4x+y-4}{x+y-1}$を解く.

    まず,次の連立方程式
    \[ \begin{cases}
        4x+y-1 = 0 \\
        x+y-1 = 0
    \end{cases}
    \iff
    \begin{pmatrix}
        4 & 1 \\
        1 & 1
    \end{pmatrix}
    \begin{pmatrix}
        x \\ y
    \end{pmatrix}
    = 
    \begin{pmatrix}
        1 \\ 1
    \end{pmatrix}\]
    を解くと,$(x,y)=(1,0)$を得る.

    ここで,$u \coloneqq x-1, \; v \coloneqq y-0$とおく.$x=u+1, \; y=v$であるから
    \[ \frac{dy}{dx} = \frac{dv}{du} = \frac{4u+v}{u+v}\]
    となり,これは同次形である.あとはガリガリ計算すると,以下を得る:
    \begin{align*}
        (y-2x+2)^3(y+2x+2) = c \quad (c \neq 0).
    \end{align*}
\end{example}


\clearpage
%第7回
\section{線形微分方程式} %第7回
\subsection{線形とは}
\begin{equation}
    y'(x) + P(x)y(x) = Q(x). \label{eq:1order-lode}
\end{equation}
$Q(x)=0$ならば,これは同次の1階線形微分方程式である.一方,$Q(x) \neq 0$ならば,これは非同次の1階線形微分方程式である.

\begin{note*}[線形とは]
    $y_1,\; y_2$を同次方程式の解とする.すなわち,次が成り立つ:
    \[ y_1' + P(x)y_1 = 0 \quad \text{and} \quad y_2' +  P(x)y_2 = 0.\]
    \textbf{線形である}とは,$y_1+y_2, \; \alpha y_1 \; (\alpha \in \mathbb{K})$が同次方程式の解であることをいう.しばしば,和と定数倍で閉じている,などという.

    「線形」という言葉はここ以外でも聞いたことがあるだろう.例えば,線形代数における線形写像は最たる例である.線形写像がピンとこない場合は線形代数を復習すること.
\end{note*}

$y_1+y_2$が解となっているか確認しよう.$(y_1+y_2)'+P(x)(y_1+y_2)=0$がなりたてばO.K.である.
\begin{align*}
    (y_1+y_2)'+P(x)(y_1+y_2) &= (y_1'+y_2')+P(x)(y_1+y_2) \\
    &= (y_1'+P(x)y_1)+(y_2'+P(x)y_2) \\
    &= 0+0 = 0
\end{align*}
となり,確かになりたつ.

$\alpha y_1 \; (\alpha \in \mathbb{R})$が解となっているか確認しよう.$(\alpha y_1)'+P(x)(\alpha y_1)=0$がなりたてばO.K.である.
\begin{align*}
    (\alpha y_1)'+P(x)(\alpha y_1) &= \alpha y_1' + \alpha P(x)y_1 \\
    &= \alpha (y_1' + P(x)y_1) \\
    &= \alpha \cdot 0 =0
\end{align*}
となり,確かになりたつ.

以上より,和と定数倍で閉じていることから,線形である.

\begin{example}[線形ではない例]
    $y''+y^2=0$について,これは線形ではない.実際,$\alpha \in \mathbb{R}$に対して$y=\alpha y_1$とおくと
    \[ \alpha y_1' + \alpha^2 {y_1}^2 = \alpha (y_1' + \alpha {y_1}^2)\]
    となり,定数倍で閉じてないことよりわかる($y_1' + \alpha {y_1}^2$が0になるかは不明だから).
\end{example}

ここで,以後の議論で混乱を招かないために,記法について断りを入れておく.$P(t)$の不定積分の変数が$x$であるとき
\[ \int^{x} P(t) \; dt\]
とかく.これは$x$の式となる.

次のような微分方程式を考えよう:
\begin{align}
    y'+P(x)y &= 0, \label{eq:h} \\
    y'+P(x)y &= Q(x). \label{eq:i}
\end{align}
まず\eqref{eq:h}に着目しよう.$y'=-P(x)y$と同値であり,これは変数分離形である.ではこれを解いてみよう.

\begin{align*}
    \frac{dy}{y} &= -P(x) \; dx \\
    \log|y| &= - \int^{x} P(t) \; dt + c 
\end{align*}
であるから,$|y|=e^ce^{-\int^{x} P(t) \; dt}$すなわち$y=Ke^{-\int^{x} P(t) \; dt}$となる($K=0$は自明解).

$e^{-\int^{x} P(t) \; dt}$が解であるから,両辺$e^{\int^{x} P(t) \; dt}$をかけると
\begin{align*}
    e^{\int^{x} P(t) \; dt}(y'+P(x)y) &= 0 \\
    e^{\int^{x} P(t) \; dt}y' + e^{\int^{x} P(t) \; dt} P(x)y &= 0 \\
    e^{\int^{x} P(t) \; dt}y' + \left(e^{\int^{x} P(t) \; dt}\right) y &= 0.
\end{align*}
積の微分法より
\[ \frac{d}{dx} \left(e^{\int^{x} P(t) \; dt}y\right)\]
となり,積分すると
\[ e^{\int^{x} P(t) \; dt}y = c\]
となり,両辺に$e^{-\int^{x} P(t) \; dt}$をかけると
\[ y = ce^{-\int^{x} P(t) \; dt}\]
を得る.

この考えは\eqref{eq:i}を解くときに役立つ.では,\eqref{eq:i}を解いてみよう.

\eqref{eq:h}の両辺に$e^{\int^{x} P(t) \; dt}$をかけると
\[ e^{\int^{x} P(t) \; dt}y' + e^{\int^{x} P(t) \; dt} P(x)y = e^{\int^{x} P(t) \; dt} Q(x)\]
となり,\eqref{eq:h}のとき同様に
\[ \frac{d}{dx} \left(e^{\int^{x} P(t) \; dt} y\right) = e^{\int^{x} P(t) \; dt} Q(x)\]
となる.この両辺を$x$で積分すると
\[ e^{\int^{x} P(t) \; dt} y(x) = \int^x e^{\int^{x} P(\tau) \; d\tau} dt + c\]
である.よって,次のような解を得る:
\begin{equation}
    y(x) = e^{-\int^{x} P(t) \; dt} \left( \int^x e^{\int^{x} P(\tau) \; d\tau} Q(t) \; dt +c \right).
\end{equation}

これを\textbf{定数変化公式}という.特に$Q \equiv 0$のとき
\[ y(x) = ce^{-\int^{x} P(t) \; dt}\]
となり,これは\eqref{eq:h}の解になっている.

\begin{note*}[「定数変化」の意味 ---どこらへんが定数なの?---]
    \eqref{eq:h}の解は$y(x) = ce^{-\int^{x} P(t) \; dt}$であった.\eqref{eq:i}の解$\tilde{y}$があったとすると
    \begin{align*}
        \tilde{y} &= y(x) = e^{-\int^{x} P(t) \; dt} \times (\text{xの関数}) \\
        &= C(x) e^{-\int^{x} P(t) \; dt}
    \end{align*}
    とかけていないといけない.ここで,$C(x)$とは$x$の関数と見る.「定数変化」というのは,\eqref{eq:h}における定数$c$を,$C(x)$という$x$の関数に置き換わったことを指す.それゆえ,定数変化といっているわけである.
\end{note*}

\begin{remark}
    $Q(x)$が決まれば$\tilde{y}$がわかる.
\end{remark}

実際,$\tilde{y}=C(x)e^{-\int^{x} P(t) \; dt}$が解だと思って\eqref{eq:i}に代入する.
\begin{align*}
    \tilde{y} + P(x)\tilde{y} &= C'(x)e^{-\int^{x} P(t) \; dt} + C(x)\{-e^{-\int^{x} P(t) \; dt} (\int^x P(t) \; dt)' \} + P(x) (C(x)e^{-\int^{x} P(t) \; dt}) \\
    &= C'(x) e^{-\int^{x} P(t) \; dt} \\
    &= Q(x).
\end{align*}
ここで,$(\int^x P(t) \; dt)' = P(x)$であることに注意しよう(微分積分学の基本定理).したがって
\begin{align*}
    C'(x) &= Q(x) e^{\int^{x} P(t) \; dt} \\
    C(x) &= \int^x  Q(t) e^{\int^{x} P(\tau) \; d\tau} \; dt + C.
\end{align*}
$C(x)$を$\tilde{y}$に代入すると
\[ \tilde{y} = e^{-\int^{x} P(t) \; dt} \left\{ \int^x Q(t) e^{\int^{x} P(\tau) \; d\tau} \; dt + C\right\}\]
を得る.

\subsection{具体例}
\begin{example}
    抵抗$R$,インダクタンス$L$,直流電流$E$を直列につないだ回路を考える.ただし,$R,L,E$は定数である.スイッチSを閉じたあと,時刻$t$での電流$I(t)$を求めたい.\footnote{図は省略させていただく.描くのは結構しんどいからである.この問題はものすごく有名であるから適当な書物にあたってほしい.なお,数理モデリングで全く同じ内容を扱っている.覚えているかな?}

    抵抗$R$での逆起電力は$RI$で,インダクタンスでは$L\frac{dI}{dt}$となるから,キルヒホッフの法則より
    \begin{equation}
        RI+L\frac{dI}{dt} = E
    \end{equation}
    という,線形の微分方程式が得られる.ここで,次の2つの微分方程式を考えよう:
    \begin{align}
        I'(t) + \frac{R}{L}I(t) &= 0, \label{eq:kairo-H} \\
        I'(t) + \frac{R}{L}I(t) &= \frac{E}{L}. \label{eq:kairo-I}
    \end{align}
    \eqref{eq:kairo-H}の解の1つとして$I(t) = e^{-\frac{R}{L}t}$がある.\eqref{eq:kairo-I}の両辺に$e^{\frac{R}{L}t}$をかけると
    \begin{align*}
        e^{\frac{R}{L}t} I'(t) + e^{\frac{R}{L}t} \frac{R}{L} I(t) &= \frac{E}{L} e^{\frac{R}{L}t} \\
        \frac{d}{dx} \left(e^{\frac{R}{L}t} I(t) \right) &= \frac{E}{L} e^{\frac{R}{L}t}
    \end{align*}
    より
    \begin{align*}
        e^{\frac{R}{L}t} I(t) &= \int \frac{R}{L} e^{\frac{R}{L}t} \; dt + c \\
        &= \frac{E}{R} e^{\frac{R}{L}t} + c.
    \end{align*}
    よって
    \[ I(t) = \frac{E}{R} + ce^{-\frac{R}{L}t}.\]
    さて,初期条件$I(0) \coloneqq 0$を課すと,$c=-\frac{E}{R}$となる.このとき
    \[ I(t) = \frac{E}{R}(1 - e^{-\frac{R}{L}t}). \]
    $t \to \infty$なる極限をとると
    \[ I(t) \to \frac{E}{R}. \]
    微分方程式から得られた結果は,直感(それとも知識?)と一致していることが確かめられたであろう.
\end{example}

\begin{homework*}
    \eqref{eq:kairo-H}の解としてある$I(t) = e^{-\frac{R}{L}t}$を用いて,定数変化法で解け.
\end{homework*}

\begin{note*}[交流電流のときは...]
    $L\frac{dI}{dt} + RI = \sin \omega t$とすればよい.解き方はひとそれぞれ.
\end{note*}

%\begin{talk*}[回路図]
    %先生の描いた図において,起電力がコンデンサーに見えた.本当は,線の長さを明確に変えるべきなのだが,どちらもほぼ同じであった.後日,そのことを物理コースの知り合いが指摘していた.実は講義中に,先生は物理の人いる?と聞いて,実際に彼女に合っているかどうかを尋ねていた.そのとき,彼女はうやむやな表情を浮かべながら正しいと主張した.そのときに「違う.それはコンデンサー.」と言えば良かったのにと,わたしは勝手に思ってしまった.なんで指摘しなかったのかはわからない.本人の性格が起こした事柄といえよう.
%\end{talk*}


\clearpage
%第8回
\section{Bernoulliの方程式・Riccatiの方程式} %第8回
\subsection{Bernoulliの方程式}
次のような方程式を考えよう:
\begin{equation}
    y' + P(x)y = Q(x)y^{\alpha}. \label{eq:ast-ber}
\end{equation}
$\alpha=0$ならば,\eqref{eq:ast-ber}は非同次線形方程式である.一方$\alpha = 1$ならば$y'+(P(x)-Q(x))y=0$となり,これは同次線形方程式である.以後,$\alpha \neq 0, 1$としよう.

ここで,$\bm{z(x)=y(x)^{1-\alpha}}$と変換すると$z'(x)=(1-\alpha)y^{-\alpha}y'$であるから,\eqref{eq:ast-ber}について両辺$y^{\alpha}$で割ると
\[ \frac{y'}{y^{\alpha}} + P(x)y^{1-\alpha} = Q(x)\]
となる.したがって
\begin{align*}
    \frac{z'(x)}{1-\alpha} + P(x)z(x) &= Q(x) \\
    \ie \quad z'(x) + (1-\alpha)P(x)z(x) &= (1-\alpha)Q(x).
\end{align*}
よって,非同次の線形方程式となるから,これは解ける($\alpha=0$のときに帰着).

\begin{example} \label{example:ber}
    $\displaystyle y'+\frac{y}{x} = x^2y^3$を解く.$P(x) = 1/x, \; Q(x) = x^2, \; \alpha=3$といえる.では,$z(x) \coloneqq y{1-3} = y^{-2}$となる.$z'=-2y^{-3}y'$であり,考えている方程式について両辺$y^3$で割ると
    \begin{align*}
        \frac{y'}{y^3} + \frac{1}{x}y^{-2} &= x^2 \\
        \quad -\frac{1}{2}z' + \frac{1}{x}z &= x^2 \\
        \ie \quad z'-\frac{z}{x} &= -2x^2 \tag{eq:ast-section8}
    \end{align*}
    を得る.ではこれを解こう.$z'-z/x= 0$の解を求めてから\eqref{eq:ast-section8}の解を求めようとするとしんどいと思われるので,適当な積分因子をかけることにより面倒さ(?)を回避しよう.

    両辺に$x^{-2}$をかけると,以下のようになる:
    \begin{align*}
        \frac{1}{x^2}z' - \frac{2}{x^3}z &= -2 \\
        \frac{d}{dx}\left(\frac{1}{x^2}z\right) &= -2 \\
    \end{align*}
    より
    \begin{align*}
        \frac{1}{x^2}z &= -2x + c \\
        z &= -2x^3+cx^2 \\
        \frac{1}{y^2} &=  -2x^3+cx^2 \\
        \therefore \quad y^2(-2x^3+cx^2) &= 1.
    \end{align*}
    ちなみに,$y \equiv 0$も解である.
\end{example}

\begin{homework*}[任意]
    次の手順で例\ref{example:ber}を解け.
    \begin{enumerate}
        \item $z'-z/x= 0$の解を求めよ.
        \item $z'-z/x = -2x^2$の解を求めよ.
    \end{enumerate}
\end{homework*}

\begin{remark}
    Bernoulliの方程式において,$\alpha>0$のときは$y'+P(x)y=Q(x)y^{\alpha}$について,$y \equiv 0$も解となっている.
\end{remark}

\subsection{Riccatiの方程式}
次のような方程式を考えよう:
\begin{equation}
    y'(x) + P(x) + Q(x)y(x) + R(x)y^2 = 0. \label{eq:riccati}
\end{equation}
$R(x) \equiv 0$ならば,\eqref{eq:riccati}は線形微分方程式である.一方,$P(x) \equiv 0$ならば,$\alpha=2$のBernoulli型である.

さて,\eqref{eq:riccati}は一般には解きにくい.幸いなことに,1つの解$y_0$がわかると,一般解を記述できる.

$y_0$を\eqref{eq:riccati}の解とすると,次がなりたつ:
\[ {y_0}' + P(x) + Q(x)y_0 + R(x){y_0}^2 = 0. \label{eq:riccati-y_0} \]
\eqref{eq:riccati}$-$\eqref{eq:riccati-y_0}を実行すると
\begin{align*}
    (y-y_0)' + (P(x)-P(x)) + Q(x)(y-y_0) + R(x)(y^2-{y_0}^2) &= 0 \\
    \ie \quad (y-y_0)' + Q(x)(y-y_0) + R(x)(y^2-{y_0}^2) &= 0.
\end{align*}
$y^2-{y_0}^2$について,こいつを上手く変形すれば,Bernoulliの方程式に帰着できそうである.実際,次のように変形すれば達成される:
\begin{align*}
    y^2-{y_0}^2 &= (y-y_0)^2 + 2yy_0 - {y_0}^2-y^2 \\
    &= (y-y_0)^2 +  2y_0(y-y_0).
\end{align*}
$u(x) \coloneqq y-y_0$とおくと
\begin{align*}
    u'(x) + Q(x)u(x) + R(x)(u^2(x)+2y_0u(x)) &= 0 \\
    \therefore \quad u'(x) + (Q(x)+2y_0)u(x) + R(x)u^2(x) &= 0.
\end{align*}
これはBernoulli型の方程式!

\begin{example}
    $y'+(2x-1)y-(x-1)y^2=x$を解く.まず$y \equiv 1$が解であることがわかる(直接代入して確認せよ).
    
    $u(x) \coloneqq y-1$とおくと$y=u(x)+1$.$y'=u'$と合わせて次を得る:
    \begin{align*}
        u'(x) + (2x-1)(u+1) - (x-1)(u'+1) &= x \\
        \ie \quad u'(x) L+ u(x) = (x-1)u^2.
    \end{align*}
    これは$\alpha=2$のBernoulli方程式であり,$\alpha>0$より,$y \equiv 0$は解の1つである.

    $u'(x) L+ u(x) = (x-1)u^2$を解こう.$z \coloneqq u^{1-2}=u^{-1}$とおく.$z'=-u^{-2}u'$であり,両辺$u^{-2}$をかけると
    \begin{align*}
        \frac{u'}{u^2} + u^{-1} &= x-1 \\
        \therefore \quad -z' + z &= x-1.
    \end{align*}
    これ以降は各自の宿題とする(授業では演習として各自が解いた記憶がある.なのでみんなも計算しよう).
\end{example}

\begin{homework*}
    好きな方法でよいから,$u'(x) + u(x) = (x-1)u^2$を解け.

    答えは$y=\frac{1}{xe^{-x}+c}+1, \; y \equiv 0, \; 1$である.
\end{homework*}


\clearpage
%第9回
\section{完全微分方程式} %第9回
次のような方程式を考えよう:
\begin{equation}
    P(x,y) + Q(x,y)y' = 0. \label{eq:perfect-ode}
\end{equation}
ただし,$P,\;Q$は$x,y$の2変数関数で既知とする.$y'=dy/dx$であったから,形式的には次のようにも捉えられる:
\begin{equation}
    P(x,y) \; dx + Q(x,y) \; dy = 0. \label{eq:perfect-ode-2}
\end{equation}

左辺が,ある2変数関数$u(x,y)$の全微分
\[ du = \frac{\partial }{\partial x}dx + \frac{\partial }{\partial y}dy \]
と表現されたとき,\eqref{eq:perfect-ode}または\eqref{eq:perfect-ode-2}を\textbf{完全微分方程式}という.いま,$du \equiv 0$の解は$u(x,y) \equiv C$となる.ここで,$C$は定数である.

さて質問.
\begin{question*}
    $u(x,y) \equiv C$の解は,\eqref{eq:perfect-ode}または\eqref{eq:perfect-ode-2}をみたすだろうか.
\end{question*}
\begin{proof}[\textbf{答え}]
    正解は\textbf{かける}.実際,以下のようにしてわかる.

    $u(x,y) \equiv C$をみたす$y$は$y=f(x)$とかけたとする.つまり,$u(x,f(x))=0$がなりたつ.両辺を$x$で微分すると
    \begin{align*}
        \frac{\partial u}{\partial x} \frac{\partial x}{\partial x} + \frac{\partial y}{\partial y} \frac{\partial y}{\partial x} &= 0 \\
        \ie \quad u_x + u_y y' &= 0.
    \end{align*}
\end{proof}

\begin{theorem}
    $P(x,y), \; Q(x,y)$は微分できるとする.このとき,次の(1),(2)は同値:
    \begin{enumerate}
        \item \eqref{eq:perfect-ode}は完全微分方程式である.
        \item $\partial P/\partial y = \partial Q/\partial x.$
    \end{enumerate}
    さらに,完全形であるとき,解は次のように表される:
    \[ \int_{x_0}^{x} P(x,y) \; dx + \int_{y_0}^{y} Q(\textcolor{red}{x_0}, y) \; dy.\]
\end{theorem}
\begin{proof}
    ($\Rightarrow$) 完全形$\partial u/\partial x = P(x,y), \; \partial u/\partial y = Q(x,y)$である.それぞれ$y,x$で偏微分すると
    \[ \frac{\partial^2 u}{\partial y \partial x} = \frac{\partial P}{\partial y}, \quad \frac{\partial^2 u}{\partial x \partial y} = \frac{\partial Q}{\partial x}\]
    より,$\partial P/\partial y = \partial Q/\partial x$である.

    ($\Leftarrow$) $u(x,y) = \int_{x_0}^{x} P(x,y) \; dx + \int_{y_0}^{y} Q(x_0,y) \; dy$とおく.微分積分学の基本定理より
    \[ \frac{\partial u}{\partial x} = \frac{\partial}{\partial x} \int_{x_0}^{x} P(x,y) \; dx = P(x,y).\]
    同様にして
    \begin{align*}
        \frac{\partial u}{\partial y} &= \frac{\partial}{\partial y} + Q(x_0,y) \\
        &= \int_{x_0}^{x} \frac{\partial P}{\partial y} \; dx + Q(x_0,y) \\
        &= \int_{x_0}^{x} \frac{\partial Q}{\partial x} \; dx + Q(x_0,y) \\
        &= \left[Q(x,y)\right]_{x_0}^x + Q(x_0,y) = Q(x,y).
    \end{align*}
    よって,$\partial P/\partial y = \partial Q/\partial x$となり,示された.
\end{proof}

今用意した$u$について,$u \equiv c$が解,すなわち
\[ \int_{x_0}^{x} P(x,y) \; dx + \int_{y_0}^{y} Q(\textcolor{red}{x_0}, y) \; dy = c\]
は解となる.

\begin{example}
    $y' = (x - y\cos x)/\sin x$を考える.この式は
    \[ (x - y\cos x) \; dx + (-\sin x) \; dy = 0\]
    と形式的に変形できる.$P(x,y)=x - y\cos x, \; Q(x,y)=-\sin x$であるから,$\partial P/\partial y = -\cos x, \; \partial Q/\partial x = -\cos x$であるから,これは完全微分方程式である.

    したがって解は$\int_{x_0}^{x} (x-y\cos x) \; dx + \int_{y_0}^{y} (-\sin x_0) \; dy = c$となる.特に,$x_0=y_0=0$となると
    \[ \int_0^x (x-y\cos x) \; dx = \frac{x^2}{2} - y\sin x = c\]
    が解となる.
\end{example}

\begin{example}
    $(2x-2y+3) \; dx + (-2x+4y+1) \; dy = 0$を解く.

    $P(x,y) \coloneqq 2x-2y+3, \; Q(x,y) \coloneqq -2x+4y+1$とおくと,$\partial P/\partial y = -2, \; \partial Q/\partial x = -2$であるから,これは完全形である.
\end{example}
\begin{homework*}
    上の例の解を求めよ.
\end{homework*}

今まで紹介した例では,たまたま微分方程式が完全形となったが,そうではないことももちろんある.しかし,そのようなときでも,適切な関数$\lambda(x,y)$をかけて
\[ \lambda(x,y)P(x,y) + \lambda(x,y)Q(x,y)y' = 0\]
が完全形となることがある.このとき,$\lambda$を\textbf{積分因子}という.

\begin{example}[$1/y^2$をかける]
    $y-xy'=0$を考える(これは変数分離形,同次形,線形でもある).$P(x,y) \coloneqq y, \; Q(x,y) \coloneqq -x$とおく.$\partial P/\partial y \neq \partial Q/\partial x$であるから,これは完全形ではない.

    さて,両辺に$y^{-2}$をかけよう.
    \[ \frac{1}{y^2}(y \; dx - x \; dy) = \frac{1}{y} \; dx - \frac{x}{y} \; dy = du.\]
    ここで,$u \coloneqq y/x$とおいた.上式は$u$の全微分であることに注意しよう.したがって,$x=cy$が解である.
\end{example}

\begin{example}[$1/xy$をかける]
    両辺に$1/xy$をかけよう.
    \[ \frac{1}{xy}(y \; dx - x \; dy) = \frac{1}{x} \; dx - \frac{1}{y} \; dy = d\left(\log|\frac{y}{x}|\right).\]
    したがって,$\log|y/x|=c$が解である.
\end{example}

積分因子として$1/y^2$をかけようが$1/xy$をかけようが,得られた結果は変わらない.\textbf{積分因子は1つに決まらないのだ}.一般に,次のことが知られている.

\begin{remark}
    1階の常微分方程式では,積分因子が存在することが知られている.
\end{remark}

\begin{example} \label{example:homework-10-1}
    $xy^2-y^3+(1-xy^2)y'=0$を考える.解を求めるまでは,宿題とする.
\end{example}

\begin{homework*}
    次の手順にしたがって,例\ref{example:homework-10-1}を解け.$P(x,y) \coloneqq xy^2-y^3, \; Q(x,y) \coloneqq 1-xy^2$とする.
    \begin{enumerate}
        \item 完全微分方程式ではないことを示せ.
        \item $\lambda(x,y) \coloneqq x^my^m, \tilde{P}(x,y) \coloneqq (xy^2-y^3)P(x,y), \tilde{Q}(x,y) \coloneqq (1-xy^2)Q(x,y)$とおく.このとき$\partial \tilde{P}/\partial y, \; \partial \tilde{Q}/\partial x$を求めよ.
        \item $\tilde{P}(x,y),\tilde{Q}(x,y)$を踏まえて,完全系であるための必要十分条件を述べよ.
        \item 積分因子$\lambda(x,y)$を求めよ.
    \end{enumerate}
\end{homework*}




\clearpage
%第10回
\section{線形微分方程式の概説}
唐突だが,次のような方程式を考えよう:
\begin{equation}
    y'_i(x) = \sum_{j=1}^{n} a_{ij}(x)y_j(x) + b_i(x) \quad (j=1,2,\ldots,n).
\end{equation}
ここで
\begin{align*}
    \bm{y} &= (y_1(x) \; y_2(x) \; \cdots \; y_n(x))^\top, \\
    A(x) &= (a_{ij}(x))_{1\leq i,j \leq n} \in M(\mathbb{R};n\times n), \\
    \bm{b}(x) &= (b_1(x) \; b_2(x) \; \cdots \; b_n(x))^\top
\end{align*}
である.$M(\mathbb{R};n\times n)$は実$n$次正方行列の集合である.このとき,方程式は以下のようにかける:
\begin{equation}
    \bm{y}' = A(x)\bm{y}(x) + \bm{b}(x). \label{eq:ast-11-1}
\end{equation}

\begin{claim*}
    $\mathscr{H} \coloneqq \{\bm{y}(x) : \bm{y}'=A\bm{y}\}$を定める.$\mathscr{H}$は線形空間となる.
\end{claim*}
\begin{proof}
    $\bm{y}_1, \bm{y}_2 \in \mathscr{H}$とする.このとき$\bm{y}_1'=A\bm{y}_1, \; \bm{y}_2'=A\bm{y}_2$である.$(\bm{y}_1+\bm{y}_2)'=\bm{y}_1'+\bm{y}_2'=A\bm{y}_1+A\bm{y}_2=A(\bm{y}_1+\bm{y}_2)$すなわち$\bm{y}_1+\bm{y}_2 \in \mathscr{H}$である.$c \in \mathbb{R}$に対して$(c\bm{y}_1)' = c \bm{y}_1' = c(A\bm{y}_1) = A(c\bm{y}_1)$すなわち$c\bm{y}_1 \in \mathscr{H}$である.

    \eqref{eq:ast-11-1}の同次形の解の集合は,和と定数倍について閉じている.このことから,主張が従う.
\end{proof}

\eqref{eq:ast-11-1}の左辺について,$\bm{f}(x,\bm{y}) \coloneqq A(x)\bm{y}(x) + \bm{b}(x)$とおく.もし,$A(x), \bm{b}(x)$が連続のとき$\bm{y}_0=\bm{c}$となる解を構成できる($\because$Picardの逐次近似法).では,他に解はないのだろうか.それを考えてみよう.

$\bm{y}(x), \bm{z}(x)$が\eqref{eq:ast-11-1}の解とする,すなわち
\begin{equation}
    \begin{cases}
        \bm{y}'(x) = A(x)\bm{y}(x) + \bm{b}(x), \quad \bm{y}(x_0) = \bm{c}, \\
        \bm{z}'(x) = A(x)\bm{z}(x) + \bm{b}(x), \quad \bm{z}(x_0) = \bm{c}
    \end{cases}
\end{equation}
をみたすとする.示すべきことは$\bm{y}(x) \equiv \bm{z}(x)$である.

$\bm{w}(x) \coloneqq \bm{y}(x) - \bm{z}(x)$とおく.$\bm{w}(x) \equiv \bm{0}$が示されればO.K.である.
\[ \bm{w}'(x) = (\bm{y}-\bm{z})' = (A\bm{y}+\bm{b})-(A\bm{z}+\bm{b}) = A(\bm{y}-\bm{z}) = A\bm{w}.\]
さらに,$\bm{y}(x_0)-\bm{z}(x_0) = \bm{c}-\bm{c}=\bm{0}$.よって,$\bm{w}$は$\bm{w}'=A\bm{w}+\bm{b}, \; \bm{w}(x_0)=\bm{0}$をみたす.

ここで,$x>x_0$として積分すると
\begin{align*}
    \int_{x_0}^{x} \bm{w}' \; dx &= \bm{w}(x) - \bm{w}(x_0) = \bm{w}, \\
    \therefore \quad \bm{w}(x) &= \int_{x_0}^{x} A(x)\bm{w}(x) \; dx.
\end{align*}

以後の議論のため,記号の定義を以下で行う.
\begin{definition}
    $\bm{a}=(a_1 \; a_2 \; \cdots \; a_n)^\top$に対して,$\|\bm{a}\|$を次で定義する:
    \[ \| \bm{a}\| \coloneqq \left(\sum_{j=1}^{n} {a_j}^2\right)^\frac{1}{2} = \sqrt{\bm{a}\cdot\bm{a}}.\]
\end{definition}

$A(x)\bm{w}(x)$について整理したい.第$i$行については
\[ \sum_{j=1}^{n} a_{ij}(x)w_j(x) = 
\begin{pmatrix}
    a_{i1} \\ \vdots \\ a_{in} 
\end{pmatrix}
\cdot
\begin{pmatrix}
    w_1(x) \\ \vdots \\ w_n(x)
\end{pmatrix}\]
より
\begin{align*}
    \| A(x)\bm{w}(x) &= \sum_{i=1}^{n} \left(\sum_{j=1}^{n} a_{ij}(x)w_j(x)\right)^2 = \sum_{i=1}^{n} (\bm{a}_i \cdot \bm{w})^2 \\
    &\leq \sum_{i=1}^{n} \|\bm{a}_i\|^2 \|\bm{w}\|^2 \quad (?-1)\\
    &= \|\bm{w}\|^2 \sum_{i=1}^{n} \| \bm{a}_i\|^2.
\end{align*}
ここで,$\bm{a}_i$は行列$A$の第$i$行の成分を縦に並べたベクトルである.

$(?-1)$の不等式評価がどうしてなりたつのかわからない人がいるかもしれないので説明する,これは\textbf{Cauchy-Schwarzの不等式}と呼ばれるものである.主張は以下の通り:
\begin{theorem}[Cauchy-Schwarzの不等式]
    $\bm{x}, \bm{y} \in \mathbb{R}^n$について,次の不等式がなりたつ:
    \[ |\bm{x} \cdot \bm{y}| \leq \| \bm{x}\| \| \bm{y}\|.\]
    ここで,$\bm{x} \cdot \bm{y}$は$\bm{x}$と$\bm{y}$の内積,$\|\bm{x}\|$は$\bm{x}$のノルムを表す.
\end{theorem}

\begin{talk*}[Cauchy-Schwarzの不等式に関する小話]
    Cauchy-Schwarzの不等式は大学受験数学においても出現する.例えば,$x^2+y^2=1 \; (x,y \in \mathbb{R})$という条件のもとで,$3x+4y$の最大値と最小値を求める問題で威力を発揮する.Cauchy-Schwarzの不等式から
    \[ -\sqrt{x^2+y^2}\sqrt{3^2+ 4^2} \leq 3x+4y \leq \sqrt{x^2+y^2}\sqrt{3^2+ 4^2}\]
    が従う.したがって,最大値は5,最小値は$-5$となる.この他にも,$x=\cos \theta, \; y=\sin \theta$を置換して三角関数の最大最小に帰着させて解く方法など,いろいろある.
\end{talk*}

さて,$\sum_{i=1}^{n} \| \bm{a}_i\|^2$について,$A(x)$は連続より,各係数の最大値と最小値が存在する(これは,有界閉区間で定義された実数値関数については,最大最小が存在するという事実に由来する).$M ^2 \coloneqq \max_{x_0 \leq t \leq x} \sum_{i=1}^{n} \|\bm{a}_i\|^2$とおくと,$x_0<t<x$について
\[ \|A(x)\bm{w}(x)\| \leq M\|\bm{w}\|\]
が従う.
\begin{align*}
    \|\bm{w}(x)\| &= \|\int_{x_0}^{x} A(x)\bm{w}(x) \; dx\| \\
    &\leq \int_{x_0}^{x} \|A(x)\bm{w}(x)\| \; dx \\
    &\leq M \int_{x_0}^{x} \|\bm{w}(x)\| \; dx.
\end{align*}
$K(x) \coloneqq \int_{x_0}^{x} \|\bm{w}(t)\| \; dt$とおくと
\[ \frac{d}{dt}K(x) = \|\bm{w}(x)\| - \|\bm{w}_0\| = \|\bm{w}(x)\|.\]
上式がなりたつのは,微分積分学の基本定理からわかる.よって,$dK/dt \leq MK(x)$となり,$K(x_0)=0$である.

$DK/dt - MK(x) \leq 0$について,両辺$e^{-Mx}$をかけると
\[ 0 \leq e^{-Mx}\frac{dK}{dt} - e^{-Mx}MK(x) = (e^{-Mx}K)'\]
となるから,$x_0$から$x$で積分すると
\begin{align*}
    0 \leq int_{x_0}^{x} (e^{-Mx}K)' \; dx &= e^{-Mx}K - e^{-Mx_0}K(x_0) \\
    &= e^{-Mx}K.
\end{align*}
一方で,$K(x)=\int_{x_0}^x \|\bm{w}(x)\| \; dx \leq 0 \quad (\because \|\bm{w}\| \leq 0).$

以上より,$0 \leq e^{-Mx}K(x) \leq 0$を得る.よって,$K(x) \equiv 0$となり$\bm{w}(x) \equiv 0 \; (x \in [x_0,x])$が従う.つまり$\bm{y} \equiv \bm{z}$となる.

\begin{theorem}
    $A(x),\; \bm{b}(x)$が連続であるとする.初期条件$\bm{y}_0=\bm{c}$をみたす解は唯1つ存在する.
\end{theorem}

\clearpage
%第11回
\section{一般論パート}
次の方程式を考えよう:
\begin{align}
    \bm{y}'(x) &= A(x)\bm{y}(x) + \bm{b}(x), \label{eq:11-I} \\
    \bm{y}'(x) &= A(x)\bm{y}(x). \label{eq:11-H}
\end{align}
どちらも$A(x),\bm{b}$が連続で,$\bm{y}$が\eqref{eq:11-I}と$\bm{y}(x_0)=\bm{c}$をみたす解は唯1つ存在する.

$\bm{v}_j \coloneqq (0 \; \cdots \; 0 \; 1 \; 0 \; \cdots 0)^\top \; (j=1,2,\ldots,n)$は縦ベクトルで,第$j$成分が1,それ以外が0であるようなものである.これを標準基底と呼ぶ.人によっては,$\bm{v}$ではなく$\bm{e}$を用いる.これを用いると
\[ \bm{c} = \alpha_1\bm{v}_1 + \cdots + \alpha_n\bm{v}_n\]
と表せる.

ここで,$\bm{y}_k(x)$を
\[ \begin{cases}
    \bm{y}'_k = A\bm{y}_k, \\
    \bm{y}_k(x_0) = \bm{v}_k
\end{cases}\]
と定める.この$\bm{y}_1,\ldots,\bm{y}_n$を用いると
\[ \bm{y}(x) = \alpha_1\bm{y}_1 + \cdots + \alpha_n\bm{y}_n\]
とかける.

$Y(x) \coloneqq (\bm{y}_1(x),\ldots,\bm{y}_n(x))$を$n\times n$型行列値関数(成分が関数),$V \coloneqq (\bm{v}_1,\ldots,\bm{v}_n)$を$n\times n$正則関数とする.

$\bm{\alpha} \coloneqq (\alpha_1 \; \cdots \; \alpha_n)^\top$とおく($\bm{v}_k$が標準規定だと$\bm{\alpha}=\bm{c}$である).$V$は正則であるから,逆行列をもち$\bm{\alpha}= V^{-1}\bm{c}=Y^{-1}(x_0)\bm{c}$である.よって解は
\begin{align*}
    \bm{y}(x) &= \alpha_1\bm{y}_1 + \cdots + \alpha_n\bm{y}_n \\
    &= (\bm{y}_1 \; \cdots \; \bm{y}_n)
    \begin{pmatrix}
        \alpha_1 \\ \vdots \\ \alpha_n
    \end{pmatrix} \\
    &= Y(x)\cdot Y^{-1}(x_o)\bm{c}
\end{align*}
となる.

\begin{definition}[基本解・レゾルベント]
    $\bm{y}_1,\ldots,\bm{y}_n$を\textbf{基本解}といい,$R(x,x_0) \coloneqq Y(x)\cdot Y^{-1}(x_0)$を\textbf{レゾルベント}という.
\end{definition}
レゾルベントを用いると,解は$bm{y}(x) = R(x,x_0)\bm{c}$となる.

\begin{definition}[ロンスキー行列・ロンスキアン]
    $n$つの$n$次元ベクトル値関数$\bm{y}_1(x),\ldots,\bm{y}_n(x)$に対して
    \[ W(x) \coloneqq (\bm{y}_1(x),\ldots,\bm{y}_n(x))\]
    を\textbf{ロンスキー行列}といい,その行列式すなわち$\det (\bm{y}_1(x),\ldots,\bm{y}_n(x))$を\textbf{ロンスキアン}という.
\end{definition}

\begin{theorem}[Liouvilleの定理]
    $\bm{y}_k(x)$は$\bm{y}'_k = A\bm{y}_k$をみたすとする.このとき,以下がなりたつ:
    \[ W(x) = W(x_0) \exp \left(\int_{x_0}^x \tr A(x) \; dx\right).\]
\end{theorem}
一般に$\exp(\cdots) \neq 0$であるから,次のことがいえる:
\begin{note*}
    ある$x_0$で$W(x_0) \neq 0.$ $\iff$ $W(x) \neq 0.$
\end{note*}

$\{\bm{y}_1,\ldots,\bm{y}_n\}$が基本解であるとする.
\[ Y(x_0) = (\bm{y}_1(x_0) \; \cdots \; \bm{y}_n(x_0)) = (\bm{v}_1 \; \cdots \; \bm{v}_n) = V(x)\]
であり,$W(x_0) = \det Y(x_0) = \det V \neq 0$とLiouvilleの定理から,$Y(x)$の逆行列は常に存在することがわかる.

\subsection{(H)}


\clearpage
%第12回
\section{一般論パート}

\clearpage
%第13回
\section{微分演算子(その1)}
次のような定数係数の線形微分方程式を考えよう:
\begin{align}
    y^{(n)}(x) + a_1y^{(n-1)}(x) + \cdots +a_{n-1}y' + a_ny(x) &= f(x), \label{eq:I-13} \\
    y^{(n)}(x) + a_1y^{(n-1)}(x) + \cdots +a_{n-1}y' + a_ny(x) &= 0. \label{eq:H-13}
\end{align}
ただし,$y$は未知関数,$f(x)$は既知であるとする.

ここで,$y' = \frac{d}{dx}y$を次のように書く:
\[ y' \coloneqq Dy \quad \ie D \coloneqq \frac{d}{dx}.\]
2階微分,3階微分,$k$階微分については以下のように書く::
\[ \frac{d^2}{dx^2} = D^2, \; \frac{d^3}{dx^3} = D^3, \; \cdots, \; \frac{d^n}{dx^n} = D^n.\]

これを踏まえると,\eqref{eq:I-13}は次のように書くことができる:
\[ (D^n+a_1D^{n-1}+\cdots+a_{n-1}D+a_n)y = f(x).\]
この$D$を\textbf{微分演算子}とよぶ.

\begin{example}
    $a,b$を定数として,$(D-a)(D-b)y$を計算しよう.
    \begin{align*}
        (D-a)\{(D-b)y\} &= (D-a)(Dy-by) \\
        &= D(y'-by) - a(y'-by) \\
        &= (y''-by') - ay' + aby \\
        &= y'' -(a+b)y' + aby \\
        &= (D^2-(a+b)D+ab)y.
    \end{align*}
    同様にして$(D-b)(D-a)y=(D^2-(b+a)D+ab)y$が導かれる.
\end{example}

\begin{remark}
    $(D-a)$と$(D-b)$の作用の順番は交換可能であり
    \[ (D-a)(D-b)y = (D-b)(D-a)y = (D^2-(a+b)D+ab)y\]
    となり,代数的に扱うことができる.
\end{remark}

微分演算子$D$に対して,sの逆演算$D^{-1}$を
\[ D^{-1} f\coloneqq \int f(x) \; dx\]
で定義する.

\begin{note*}
    \begin{align*}
        D^{-1}(Df) &= f + C, \\
        D(D^{-1}f) &= f
    \end{align*}
    となる.ただし,$C$は定数である.このようにして,$D^{-1}D, DD^{-1}$は少し異なる.ちなみに,$DD^{-1}$を\textbf{恒等演算子}と呼ぶこともある.
\end{note*}

さて,微分演算子$D$の導入により,我々\eqref{eq:I-13}を
\[ (D^n+a_1D^{n-1}+\cdots+a_{n-1}D+a_n)y = f(x)\]
のように書くことに成功した.$L(D) \coloneqq D^n+a_1D^{n-1}+\cdots+a_{n-1}D+a_n$とおこう.\eqref{eq:I-13}は$L(D)=f(x)$のように書き直すことができる.$L(D)^{-1}$が上手に定義できれば,解$y$がわかりそうである.

\begin{example}
    $D^2y = f$について,$(D^2)^{-1}f \coloneqq \iint f(x) \; dxdx$と定義すればよい.以後,$(D^k)^{-1}$を$D^{-k}$と表すこととする.

    ちなみに,$\iint$は$x$について2回積分してくださいということを意味していて,重積分の意味ではないことに注意されたい.
\end{example}

\begin{example}
    $(D-a)y=f$を考える.変形すると$y'-ay=f$となり,これは線形である.両辺に$e^{-ax}$をかけると
    \begin{align*}
        e^{-ax}y' -e^{-ax}y &= e^{-ax}f \\
        (e^{-ax})' &= e^{-ax}f(x).
    \end{align*}
    したがって
    \begin{align*}
        e^{-ax}y &= \int e^{-ax} f(x) \; dx \\
        y &= e^{ax} \int ^{-ax} f(x) \; dx.
    \end{align*}
    この考察から,我々は$(D-a)^{-1}$を次で定義する:
    \[ (D-a)^{-1} \coloneqq e^{ax} \int e^{-ax}f(x) \; dx.\]
\end{example}

\begin{example}
    $(D-a)^2y = f(x)$を考える.$((D-a)^2)^{-1}f(x)$が知りたい.先ほどの議論より$((D-a)^2)^{-1} = (D-a)^{-2}$とかける.$z(x) \coloneqq (D-a)y$とおくと,$(D-a)z=f(x)$となる.したがって
    \[ z(x) = e^{ax} \int e^{-ax} z(x) \; dx.\]
    さらに
    \begin{align*}
        y &= (D-a)^{-1} z(x) \\
        &= e^{ax} \int e^{-ax} z(x) \; dx \\
        &= e^{ax} \int e^{-ax} \left(e^{ax} \int e^{-ax} f(x) \; dx\right) \; dx \\
        &= e^{ax} \iint e^{-ax} f(x) \; dx.
    \end{align*}
    先ほども述べたが,$\iint$は2重積分の意味ではない.この考察から
    \[ (D-a)^{-2} \coloneqq e^{ax} \iint e^{-ax} f(x) \; dx\]
    と定義すればよい.

    同様に,$(D-a)^{-k}y=f$については
    \[ (D-a)^{-k} \coloneqq e^{ax} \iint \cdots \int e^{-ax} f(x) \; dx\]
    と定義すればよい.
\end{example}

\begin{example}
    $(D-a)(D-b)y=f(x)$を考える.$((D-a)(D-b))^{-1}f$が知りたい.$z(x) \coloneqq (D-b)y$とおくと,$y=(D-b)^{-1}z(x)$となり,$y=(D-b)^{-1}z(x)$を得る.したがって,$y=(D-b)^{-1}(D-a)^{-1}f$となる.よって,次の関係を得る:
    \[ ((D-a)(D-b))^{-1} = (D-b)^{-1}(D-a)^{-1}.\]
    一方,$(D-a)(D-b)=(D-b)(D-a)$であるから
    \[ (D-b)^{-1}(D-a)^{-1} = (D-a)^{-1}(D-b)^{-1}\]
    となる.
\end{example}

以上をまとめ,$L(D)$を因数分解すると
\[ L(\lambda) = (\lambda - \lambda_1)^{n_1} ((\lambda - \lambda_2)^{n_2}) \cdots (\lambda - \lambda_r)^{n_r}\]
となる.ただし,$n_k$は$\lambda_k$の重複度,$i \neq j$のとき$\lambda_i \neq \lambda_j$,$n_1+n_2+\cdots+n_r=n$とする.このとき,$(D-\lambda_1)^{-n_1} (D-\lambda_2)^{-n_2} \cdots (D-\lambda_r)^{-n_r}f$は$L(D)y=f$という\eqref{eq:I-13}の1つの解となる.$L(D)y=f$の一般解は,基本解の一次結合にこれを加えたものになる.

$L(D)y=0$の基本解は,$(D-\lambda_j)^{-n_j}y=0$をみたすものを探せばよい.このとき,$\lambda_j$つの以下の解がでてくる:
\[ e^{\lambda_j x}, \; xe^{\lambda_j x}, \; x^2e^{\lambda_j x}, \; \ldots, \; x^{\lambda_j-1}e^{\lambda_j x}.\]

$L(D)y=0$の基本解は,次で表せる:
\[ y(x) = \sum_{j=1}^{r} \sum_{k}^{n_j-1} C_{jk} x^k e^{\lambda_j x}.\]

\begin{definition}
    \eqref{eq:I-13}:$L(D)y=f$の一般解は以下で書ける:
    \[ y = \sum_{j=1}^{r} \sum_{k}^{n_j-1} C_{jk} x^k e^{\lambda_j x} + (D-\lambda_1)^{-n_1} (D-\lambda_2)^{-n_2} \cdots (D-\lambda_r)^{-n_r}f.\]
    ここで,$(D-\lambda_j)^{-n_j}f$は次のようになる:
    \[ (D-\lambda_j)^{-n_j}f = e^{\lambda_j x} \iint \cdots \int e^{-\lambda_j x} f(x) \; dx.\]
\end{definition}

\begin{example}
    $y'''-y''-y'+y=xe^{2x}$を考える.微分演算子を用いると
    \[ (D^3-D^2-D+1)y = xe^{2x}\]
    と書き直せる.ここで,$L(D) \coloneqq D^3-D^2-D+1$とおくと,特性多項式$L(\lambda)$は$L(\lambda)=\lambda^3-\lambda^2+\lambda+1$となる.これを解くと,$\lambda=-1, 1$を得るから,一般解は次のようになる:
    \[ y(x) = c_1e^{-x} + c_2e^x + c_3xe^x + (D-1)^{-2}(D+1)^{-1}xe^{2x}.\]

    特殊解を求めよう.講義では4つの方法を紹介した.この第13回では,3つの方法を紹介する.
\end{example}

\begin{description}
    \item[(1) 定義通りにとく] まず
    \begin{align*}
        (D+1)^{-1}xe^{2x} &= e^{-x} \int e^x xe^{2x} \; dx \\
        &= e^{-x} \int x e^{3x} \; dx \\
        &= e^{-x} \left[\frac{x}{3}e^{3x}\right] - e^{-x} \int \frac{1}{3}e^{3x} \; dx \\
        &= \frac{1}{3}xe^{2x} - \frac{1}{9}e^{2x} \\
        &= \frac{1}{9}(3x-1)e^{2x}.
    \end{align*}
    つぎに
    \begin{align*}
        (D-1)^{-2}(D+1)^{-1}xe^{2x} &= (D-1)^{-2} \left\{\frac{1}{9}(3x-1)e^{2x}\right\} \\
        &= \frac{1}{9}e^x \iint e^{-x} (3x-1) e^{2x} \; dxdx \\
        &= \frac{1}{9}e^x \iint (3x-1)e^x \; dxdx \\
        &= \frac{1}{9}e^x \left(\int [(3x-1)e^x] - 3\int e^x \; dx \right) \\
        &= \frac{1}{9}e^x \int (3x-4)e^x \; dx \\
        &= \frac{1}{9}e^x \left[(3x-4)e^x - 3 \int e^x \; dx\right] \\
        &= \frac{1}{9}e^x \left[(3x-7)e^x\right] \\
        &= \frac{1}{9}(3x-7)e^{2x}.
    \end{align*}
    よって,$(D-1)^{-2}(D+1)^{-1}xe^{2x} = \frac{1}{9}(3x-7)e^{2x}$を得る.

    これは大変!
    
    \item[(2) 推移法則を使う] まず,主張は以下の通りである:
    \begin{theorem}[推移法則]
        $\lambda$を定数とするとき,次がなりたつ:
        \[ L(D)^{-1}(e^{\lambda x}f(x)) = e^{\lambda x} L(D+\lambda)^{-1} f(x).\]
    \end{theorem}
    \begin{align*}
        (D+1)^{-1}xe^{2x} &= e^{2x} L(D+3)^{-1} x \\
        &= e^{2x} \cdot e^{-3x} \int x e^{3x} \; dx \\
        &= e^x \left[\frac{1}{3}xe^{3x} - \frac{1}{3}\int e^{3x} \; dx\right] \\
        &= \frac{1}{9}e^{2x}(3x-1), \\
        (D-1)^{-2}(D+1)^{-1}xe^{2x} &= (D-1)^{-2} \left(\frac{1}{9}e^{2x}(3x-1)\right) \\
        &= e^{2x} (D+1)^{-2} \left(\frac{1}{9}(3x-1)\right) \\
        &= \frac{1}{9}(3x-7)e^{2x}
    \end{align*}
    となり,たしかに(1)の結果と一致する.

    \item[(3) 部分分数分解を用いる] $1/L(\lambda)$が次のように変形できたとしよう:
    \[ \frac{1}{L(\lambda)} = \frac{A_1}{(\lambda-\mu_1)^{n_1}} + \cdots + \frac{A_k}{(\lambda-\mu_k)^{n_k}}.\]
    $L(D)^{-1} = A_1(\lambda-\mu_1)^{n_1} + \cdots + A_k(\lambda-\mu_k)^{n_k}$となる.

    今回の場合,$L(\lambda)=(\lambda-1)^2(\lambda+1)$だから
    \[ L(D)^{-1} = \frac{1}{4}(D+1)^{-1} - \frac{1}{4}(D-1)^{-1} + \frac{1}{2}(D-1)^{-2}\]
    となる.あとは,計算.
\end{description}

\begin{homework*}
    省略された計算過程を自力で補え.
\end{homework*}

次回はMaclaurineの方法を述べる.


\clearpage
%第14回
\section{微分演算子(その2)・行列の指数関数(その1)}
\subsection{Maclaurineの方法}
次回述べなかったMaclaurineの方法を述べる.

$L(\lambda) = \lambda^p M(\lambda), \; M(\lambda) \neq 0$とする.$M(\lambda)$のMaclaurine展開を用いて$L(\lambda)^{-1}$が
\[ L(\lambda)^{-1} = \lambda^{-p} (b_0+b_1\lambda+b_2\lambda^2+\cdots)\]
とかけたとする.そして,$f(x)$が$k$次以下の整式でかけたとする.このとき
\[ L(D)^{-1}f = (b_0+b_1\lambda+b_2\lambda^2+\cdots+b_kD^k)f(x).\]

前回の例題において,$f(x)=xe^{2x}$であったが,これは明らかに整式ではない.したがって今回の方法が使えないように見える.さて,
\begin{align*}
    (D+1)^{-1} xe^{2x} &\beceq{\text{推移法則}} e^{2x} (D+1+2)^{-1} x \\
    &= e^{2x} (D+3)^{-1}x,
\end{align*}
という式に着目しよう.$xe^{2x}$を直接マクローリン展開するのではなく,「$(D+3)^{-1}$」を強引にマクローリン展開するのである.

今,$(\lambda+3)^{-1} = \frac{1}{3}\frac{1}{1+\lambda/3} = \frac{1}{3}(1-\frac{\lambda}{3}+(\frac{\lambda}{3})^2 + \cdots)$となるから
\begin{align*}
    (D+1)^{-1}xe^{2x} &= e^{2x}(D+3)^{-1}x \\
    &= e^{2x} \cdot \frac{1}{3}(1-\frac{D}{3}+\frac{D^2}{9}-\cdots)x \\
    &= e^{2x} \cdot \frac{1}{3}(x-\frac{1}{3}) \\
    &= \frac{1}{9}(3x-1)e^{2x}.
\end{align*}

次に,$(D-1)^2 \left(e^{2x}\cdot \frac{3x-1}{9}\right)$を計算する.
\begin{align*}
    (D-1)^2 \left(e^{2x}\cdot \frac{3x-1}{9}\right) \beceq{\text{推移法則}} \frac{1}{9}e^{2x}(D+1)^{-2}(3x-1).
\end{align*}
$(\lambda+ 1)^{-2}$を知るためには
\begin{align*}
    (\lambda+1)^{-2} &= \left(\frac{1}{\lambda+1}\right)^2 \\
    &= (1-\lambda+\lambda^2-\lambda^3+\cdots)^2
\end{align*}
を計算すればよい.また,$\frac{d}{d\lambda}(1+\lambda)^{-1} = -(1+\lambda)^{-2}$より
\begin{align*}
    (1+\lambda)^{-2} &= -\frac{d}{d\lambda}(1+\lambda)^{-1} \\
    &= -\frac{d}{d\lambda}(1-\lambda+\lambda^2-\lambda^3+\cdots) \\
    &= 1-2\lambda+ 3\lambda^2-4\lambda^3+\cdots.
\end{align*}
ちなみに,直接$(1-\lambda+\lambda^2-\lambda^3+\cdots)^2$を計算してもよい.よって
\begin{align*}
    (D+1)^{-2}(3x-1) &= (1-2D)(3x-1) \\
    &= 3x-7.
\end{align*}
となり,$y=\frac{1}{9}e^{2x}(3x-7)$を得る.

\begin{note*}[なんで$(1-2D)$で打ち切ったの?]
    $f(x)=3x-1$は高々1次であるから,$D^2$以降を作用させたとしても,その計算結果は0となる.したがって,1次までで十分である.
\end{note*}

別解答としては,次のようなものが挙げられる:
\begin{align*}
    (D+1)^{-2} (3x-1) &= (D+1)^{-1} (D+1)^{-1} (3x-1) \\
    &\beceq{\text{Maclaurine}} (D+1)^{-1} \left(\textcolor{red}{(1-D)}(3x-1)\right) \\
    &= (D+1)^{-1}(3x-4) \\
    &\beceq{\text{Maclaurine}} \textcolor{red}{(1-D)}(3x-4) \\
    &= 3x-7.
\end{align*}
いずれの方法にせよ,計算できるようにすることが大事である.テストで必ず出題される.


\subsection{定数係数の連立微分方程式について}
次のような連立微分方程式を考えたい:
\begin{equation} \label{eq:14-連立}
    \begin{cases}
        y_1' = 3y_1 + 4y_2 \\
        t'_2 = 2y_1 + y_2.
    \end{cases}
\end{equation}
$\bm{y} \coloneqq (y_1 \; y_2)^\top$とすれば,\eqref{eq:14-連立}は
\[ 
\begin{pmatrix}
    y_1 \\ y_2
\end{pmatrix}
= 
\begin{pmatrix}
    3 & 4 \\ 1 & 1
\end{pmatrix}
\begin{pmatrix}
    y_1 \\ y_2
\end{pmatrix}.\]
のように書ける.$A \coloneqq
\begin{pmatrix}
    3 & 4 \\ 1 & 1
\end{pmatrix},$とおけば,$\bm{y}' = A\bm{y}$となる.

復習をしよう.$y'=ay$の解は$y=ce~{ax}$であった.今回は$\bm{y}' = A\bm{y}$と形は似ているが,係数は行列だし,$y$はベクトルである.ここで,次のような疑問が浮かんでくるのは自然であろう:
\begin{question*}
    $e^{ax}$に相当するもの,つまり$\bm{e^{xA}}$なるものは存在するのか.
\end{question*}
ちなみに,$\bm{e^{xA}}$を\textbf{行列の指数関数}という.

次の微分方程式を考えよう:
\begin{equation}
    \bm{y}' = A\bm{y}. \label{eq:14-H}
\end{equation}
ただし,$\bm{y} \coloneqq (y_1(x) \; y_2(x) \; \cdots \; y_n(x))^\top$,$A=(a_{ij})$を定数行列とする.\eqref{eq:14-H}は次のようにも書ける:
\begin{equation}
    D\bm{y} = A\bm{y}.
\end{equation}
今,$\bm{y}=e^{\lambda x}\bm{c}$が解だと思って\eqref{eq:14-H}に代入してみよう.
\begin{align*}
    D\bm{y} &= \lambda e^{\lambda x} \bm{c}, \\
    A\bm{y} &= e^{\lambda x} \cdot A\bm{c}.
\end{align*}
したがって
\begin{align*}
    \lambda e^{\lambda x} \bm{c} &= A(e^{\lambda x} \bm{c}) \\
    (A-\lambda I)(e^{\lambda x}\bm{c}) &= \bm{0} \\
    e^{\lambda x}(A-\lambda)\bm{c} &= \bm{0}.
\end{align*}
ここで,$I$は単位行列である.
\begin{remark}
    \eqref{eq:14-H}の自明でない解を見つけるためには,$(A-\lambda)\bm{c} = \bm{0}$をみたす$\bm{c} \neq \bm{0}$を探せばよい.
\end{remark}
このとき,$\lambda$を固有値,$\lambda$に対応する$\bm{c}$を固有ベクトルという.$(A-\lambda)\bm{c} = \bm{0}$が$\bm{c} \neq \bm{0}$なる解をもつための必要十分条件は$\det (A-\lambda) = 0$である.このことがスッとこない場合は,線形代数を復習すること.
\begin{note*}
    $\det (A-\lambda I) \neq 0$だとすると,逆行列$(A-\lambda I)^{-1}$が存在するから
    \[ (A-\lambda I)^{-1} (A-\lambda I) \bm{c} = \bm{c} = \bm{0}\]
    となり,$\bm{c} = \bm{0}$という自明な解が得られてしまう.
\end{note*}

さて,\eqref{eq:14-H}の初期値問題
\[ \begin{cases}
    \bm{y}' = A\bm{y}, \\
    \bm{y}(0) = \bm{\alpha}
\end{cases}\]
を考えよう.ここで,指数関数のマクローリン展開について復習しておこう.
\begin{remind}[指数関数の]
    $e^{ax}$のMaclaurine展開は次である:
    \begin{align*}
        e^{ax} &= 1 + ax + \frac{(ax)}{2!} + \frac{(ax)^3}{3!} + \cdots \\
        &= \sum_{n=0}^{\infty} \frac{x^k}{k!} a^k.
    \end{align*}
\end{remind}
形式的に,次のようにおいてみる.つまり,$a$を行列$A$で置き換えてみる.
\begin{align*}
    e^{xA} &\coloneqq I + xA + \frac{1}{2!}(xA)^2 + \cdots + \frac{1}{n!}(xA)^n + \cdots \\ 
    &= \sum_{k=0}^{\infty} \frac{x^k}{k!}(A^k).
\end{align*}
これを\textbf{行列の指数関数}と定義し,$e^{xA}$とかく.
\begin{note*}
    右辺の級数は,"良い"収束をする.要は,一様収束のこと.ここらへんについては\ref{book:takahashi}などに書いてある.
\end{note*}
$A^k$が$n$次正方行列であるから,右辺の級数も同様である.つまり$e^{xA}$は$n$次正方行列である.

ここで,$e^{xA}$の性質について述べよう.
\begin{proposition}
    行列の指数関数$e^{xA}$は次のような性質をもつ:
    \begin{enumerate}[label=(\arabic*)]
        \item $e^{x O} = I$.ここで$O$は零行列である.
        \item $(e^{xA})^{-1} = e^{-xA}.$
        \item $e^{(x+y)A} = e^{xA} \cdot e^{xA} = e^{yA} \cdot e^{xA}.$
        \item $e^{x(aI+A)} = e^{ax} e^{xA}.$
        \item $\displaystyle \frac{d}{dx}(e^{xA}) = A e^{xA}.$
    \end{enumerate}
\end{proposition}
今,$e^{xA}$に初期値$\bm{\alpha}$を作用させた$\bm{y}(x)=e^{xA}\bm{\alpha}$について
\begin{align*}
    \bm{y}(0) &= e^{0A} \cdot \bm{\alpha} = e^0 \pmb{\alpha} = \bm{\alpha}, \\
    \bm{y}(x) &= \frac{d}{dx}(e^{xA}\bm{\alpha}) = Ae^{xA}\bm{\alpha} = A\bm{y}(x),
\end{align*}
つまり$\bm{y}(0)=\bm{\alpha}, \; \bm{y}(x)=A\bm{y}(x)$であるから,$\bm{y}(x)=e^{xA}\bm{\alpha}$が初期値問題の解になっている.

\begin{theorem}
    次の初期値問題の解は,$\bm{y}(x)=e^{xA}\alpha$となる:
    \[ \begin{cases}
        \bm{y}' = A\bm{y}, \\
        \bm{y}(0) = \bm{\alpha}.
    \end{cases}\]
    存在定理から,一意性も保証されている.
\end{theorem}
さらに,非同次問題についても考えよう.
\begin{equation} \label{eq:14-I}
    \begin{cases}
        \bm{y}(x) = A\bm{y} + \bm{b}, \\
        \bm{y}(0) = \bm{\alpha}.
    \end{cases}
\end{equation}

\begin{theorem}
    \eqref{eq:14-I}の解は,次となる:
    \[ \bm{y}(x) = e^{xA}\bm{\alpha} + \int_{0}^{x} e^{(x-t)A}\bm{b}(t) \; dt.\]
    第1項目は基本解部分(余因子),第2項目は特殊解となる.
\end{theorem}

\vfill
次回予告をしておこう.
\begin{question*}
    行列$A$に対する,$e^{xA}$は何者なのか.そしてそれはいかにして求めるのか.
\end{question*}


\clearpage
%第15回
\section{行列の指数関数(その2)}
念のため,前回までの復習をしよう.\eqref{eq:14-I}の解は,次となることを確かめた:
    \[ \bm{y}(x) = e^{xA}\bm{\alpha} + \int_{0}^{x} e^{(x-t)A}\bm{b}(t) \; dt.\]
前回を通して一貫した疑問に「$e^{xA}$は何者なのか」が挙げられる.今回はこれについて答えを与えよう.

以後の議論では,
\begin{center}
    ($\ast$) 行列$A$に対して,ある正則行列$P$があって$P^{-1}AP$が「簡単」になるとき
\end{center}
を考えることとする.

\begin{note*}[「簡単」とは]
    「簡単」とは,$A$の固有値を求めたとき,各固有値に対する固有ベクトルが合わせて$n$個あることを指す.

    ちなみに,$P^{-1}AP$が対角行列にならない場合がある.いわゆる\textbf{ジョルダン標準形}になるときである.これも含めた議論についてはいろいろな本に書いてある.わたしは\ref{book:takahashi}でそれをみたとき,そっと本を閉じた.読む気にもなれなかった.ただ一度は読まねばならない.
\end{note*}
\begin{talk*}
    中川先生も,ジョルダン標準形も含めた議論を初めて見たとき,嫌になって本を閉じたらしい.
\end{talk*}

($\ast$)の意味について考察しよう.$A$の固有値$\lambda_1 \leq \lambda_2 \leq \cdots \leq \lambda_n$に対して,固有ベクトルをそれぞれ$\bm{u}_1,\; \bm{u}_2, \ldots, \bm{u}_n$とおく.正則行列$P$を次で定めよう:
\[ P \coloneqq (\bm{u}_1 \; \bm{u}_2 \; \cdots \; \bm{u}_n)^\top. \]
ただし,$P$は実$n \times n$正方行列である.このとき,$P$は正則で逆行列$P^{-1}$が存在する.
したがって
\[ B = P^{-1}AP = 
\begin{pmatrix}
    \lambda_1 & & & \text{\Large O} \\
     & \lambda_2 & & \\
     & & \ddots & \\
     \text{\Large O}& & & \lambda_n
\end{pmatrix}\]
となる.対角成分以外は0であることに注意しよう(これは($\ast$)という場合を考えているから).

さて,行列の指数関数$e^{xA}$は次で定義されていた:
\[ e^{xA} \coloneqq 1 + xA + \frac{1}{2!}(xA)^2 + \cdots + \frac{1}{n!}(xA)^n + \cdots. \]
$A^n$をみるために,$B=P^{-1}AP$を用いる.$A=PBP^{-1}$より,$A^n=(PBP^{-1})^n$である.
\begin{align*}
    A^n &= (PBP^{-1})(PBP^{-1}) \cdots (PBP^{-1}) \\
    &= PB^nP^{-1}.
\end{align*}
よって,$A^n=PB^nP^{-1}$を用いて
\begin{align*}
    e^{xA} &= PP^{-1} + xPBP^{-1} + \frac{x^2}{2!}(PB^2P^{-1}) + \cdots + \frac{1}{n!}(PB^nP^{-1}) + \cdots \\
    &= P(I + xB + \frac{x^2}{2!}B^2 + \cdots + \frac{x^n}{n!}B^n + \cdots)P^{-1}
\end{align*}
とできる.以上より,$e^{xA}=Pe^{xB}P^{-1}$を得る.

$e^{xB}$に着目しよう.
\begin{align*}
    e^{xB} &= 
    \begin{pmatrix}
        1+x\lambda_1+\frac{(x\lambda_1)^2}{2!}+\cdots & 0 & \cdots & 0 \\
        0 & 1+x\lambda_2+\frac{(x\lambda_2)^2}{2!}+\cdots & \cdots & 0 \\
        \vdots & \vdots& \ddots & \vdots \\
        0 & 0 & \cdots & 1+x\lambda_n+\frac{(x\lambda_n)^2}{2!}+\cdots
    \end{pmatrix} \\
    &= 
    \begin{pmatrix}
        e^{\lambda_1x} & 0 & \cdots & 0 \\
        0 & e^{\lambda_2x} & \cdots & 0 \\
        \vdots & \vdots & \ddots & \vdots \\
        0 & 0 & \cdots & e^{\lambda_nx}
    \end{pmatrix}
\end{align*}
より,$e^{xB}$が対角行列として表現できたから,$e^{xA} = Pe^{xB}P^{-1}$として計算すればよい.

\begin{example}
    $A \coloneqq \begin{pmatrix} -1 & 2 \\ -4 & 5 \end{pmatrix}$とするとき,$e^{xA}$と$\bm{y}'=A\bm{y}$の一般解,そして初期値問題$\bm{y}(0)=\begin{pmatrix} 1 \\ 3 \end{pmatrix}$に対する解を求める.

    固有方程式$\phi(t)=\det (A-tI)=0$の解を求める.$\phi(t)=(t-1)(t-3)$であるから,$t=1,3$が解である.

    固有値1に対する固有ベクトルを求めよう.$\bm{y} \coloneqq (x\; y)^\top$とする.
    \[ \begin{pmatrix}
        -1-1 & 2 \\
        -4 & 5-1
    \end{pmatrix}
    \begin{pmatrix}
        x \\ y
    \end{pmatrix}
    =\bm{0}\]
    をとく.計算の末,固有ベクトルとして$(x\; y)^\top=(1 \; 1)^\top$がとれることがわかる.

    固有値3に対する固有ベクトルを求めよう.$\bm{y} \coloneqq (x\; y)^\top$とする.
    \[ \begin{pmatrix}
        -1-3 & 2 \\
        -4 & 5-3
    \end{pmatrix}
    \begin{pmatrix}
        x \\ y
    \end{pmatrix}
    =\bm{0}\]
    をとく.計算の末,固有ベクトルとして$(x\; y)^\top=(1 \; 2)^\top$がとれることがわかる.

    $\bm{u}_1 \coloneqq (1\; 1)^\top, \; \bm{u}_2 \coloneqq (1\; 2)^\top$として,$P^{-1} \coloneqq \begin{pmatrix} 1 & 1 \\ 1 & 2\end{pmatrix}$とおく.このとき,$P^{-1}=\begin{pmatrix} 2 & -1 \\ -1 & 1 \end{pmatrix}$となる.

    さて,$B \coloneqq P^{-1}AP = \begin{pmatrix} 1 & 0 \\ 0 & 3 \end{pmatrix}$とおく.ここで
    \[ e^{xB} = \begin{pmatrix} e^x & 0 \\ 0 & e^{3x} \end{pmatrix} \]
    であるから
    \begin{align*}
        e^{xA} &= Pe^{xB}P^{-1} \\
        &= 
        \begin{pmatrix}
            1 & 1 \\
            1 & 2
        \end{pmatrix}
        \begin{pmatrix}
            e^x & 0 \\
            0 & e^{3x}
        \end{pmatrix}
        \begin{pmatrix}
            2 & -1 \\
            -1 & 1
        \end{pmatrix} \\
        &= 
        \begin{pmatrix}
            2e^x-e^{3x} & -e^x+e^{3x} \\
            2e^x-2e^{3x} & -e^x+2e^{3x}
        \end{pmatrix}
    \end{align*}
    を得る.よって,一般解は,$\bm{c} \coloneqq (c_1 \; c_2)^\top$として
    \[ \bm{y}(x) = e^{xA}\bm{c}\]
    となる.初期値問題については
    \begin{align*}
        \bm{y}(x) &= 
        \begin{pmatrix}
            2e^x-e^{3x} & -e^x+e^{3x} \\
            2e^x-2e^{3x} & -e^x+2e^{3x}
        \end{pmatrix}
        \begin{pmatrix}
            1 \\ 3
        \end{pmatrix} \\
        &= -e^x\begin{pmatrix} 1 \\ 1 \end{pmatrix} + 2e^{3x} \begin{pmatrix} 1 \\ 2 \end{pmatrix} \\
        &= -e^x \bm{u}_1 + 3e^{3x} \bm{u}_2
    \end{align*}
    となる.
\end{example}

\begin{remark}
    $-e^x \bm{u}_1 + 3e^{3x} \bm{u}_2$をみればわかるように,解が$\bm{u}_1, \; \bm{u}_2$という行列$A$の固有ベクトルで表されている.
\end{remark}

\begin{homework*}
    $A \coloneqq \begin{pmatrix}
        2 & -1 \\ 
        -2 & 3
    \end{pmatrix}$だとどうなるか,各自計算せよ.
\end{homework*}






\end{document}