\section{放射性崩壊と数理モデル} %第2回
\subsection{1階微分方程式}
放射性崩壊の法則は「放射性物質の崩壊速度は,そのとき存在する物質量に比例する」ということを主張する.これは微分方程式で表せば,以下のように記述される:
\begin{equation}
    \frac{dN}{dt} = -\lambda N.
\end{equation}

$N(t)$を時刻$t$における原子数,$N(t+\Delta t)$を時刻$t+\Delta t$における原子数とする.このとき,$N(t)-N(t+\Delta t)$は$\Delta t$の時間で減少した原子数を意味する.一方で,法則より,この数は$\lambda N(t)\Delta t$とかける.したがって以下を得る:
\begin{align*}
    N(t)-N(t+\Delta t) &= \lambda N(t)\Delta t \\
    \frac{N(t+\Delta t)-N(t)}{\Delta t} &= -\lambda N(t).
\end{align*}
$\Delta t \to 0$とすると,$Dn/dt =-\lambda N$となる(cf: E.\;Ruthorford et al. (1903) \footnote{Ruthorfordは「ラザフォード」とよむ,})

ここで質問をしよう.$dN/dt = -\lambda N$はどう解けばよいのだろうか.答えは以下の通り:
\begin{proof}[\textbf{答え}]
    $\frac{1}{N} \; dN = -\lambda \; dt$であるから,両辺を積分して$\log|N| = -\lambda t + C$.したがって$N = Ae^{-\lambda t}$となる.ここで$A \coloneqq e^{\pm C}$とおいた.
\end{proof}