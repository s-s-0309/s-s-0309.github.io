\section{一般論パート}
次の方程式を考えよう:
\begin{align}
    \bm{y}'(x) &= A(x)\bm{y}(x) + \bm{b}(x), \label{eq:11-I} \\
    \bm{y}'(x) &= A(x)\bm{y}(x). \label{eq:11-H}
\end{align}
どちらも$A(x),\bm{b}$が連続で,$\bm{y}$が\eqref{eq:11-I}と$\bm{y}(x_0)=\bm{c}$をみたす解は唯1つ存在する.

$\bm{v}_j \coloneqq (0 \; \cdots \; 0 \; 1 \; 0 \; \cdots 0)^\top \; (j=1,2,\ldots,n)$は縦ベクトルで,第$j$成分が1,それ以外が0であるようなものである.これを標準基底と呼ぶ.人によっては,$\bm{v}$ではなく$\bm{e}$を用いる.これを用いると
\[ \bm{c} = \alpha_1\bm{v}_1 + \cdots + \alpha_n\bm{v}_n\]
と表せる.

ここで,$\bm{y}_k(x)$を
\[ \begin{cases}
    \bm{y}'_k = A\bm{y}_k, \\
    \bm{y}_k(x_0) = \bm{v}_k
\end{cases}\]
と定める.この$\bm{y}_1,\ldots,\bm{y}_n$を用いると
\[ \bm{y}(x) = \alpha_1\bm{y}_1 + \cdots + \alpha_n\bm{y}_n\]
とかける.

$Y(x) \coloneqq (\bm{y}_1(x),\ldots,\bm{y}_n(x))$を$n\times n$型行列値関数(成分が関数),$V \coloneqq (\bm{v}_1,\ldots,\bm{v}_n)$を$n\times n$正則関数とする.

$\bm{\alpha} \coloneqq (\alpha_1 \; \cdots \; \alpha_n)^\top$とおく($\bm{v}_k$が標準規定だと$\bm{\alpha}=\bm{c}$である).$V$は正則であるから,逆行列をもち$\bm{\alpha}= V^{-1}\bm{c}=Y^{-1}(x_0)\bm{c}$である.よって解は
\begin{align*}
    \bm{y}(x) &= \alpha_1\bm{y}_1 + \cdots + \alpha_n\bm{y}_n \\
    &= (\bm{y}_1 \; \cdots \; \bm{y}_n)
    \begin{pmatrix}
        \alpha_1 \\ \vdots \\ \alpha_n
    \end{pmatrix} \\
    &= Y(x)\cdot Y^{-1}(x_o)\bm{c}
\end{align*}
となる.

\begin{definition}[基本解・レゾルベント]
    $\bm{y}_1,\ldots,\bm{y}_n$を\textbf{基本解}といい,$R(x,x_0) \coloneqq Y(x)\cdot Y^{-1}(x_0)$を\textbf{レゾルベント}という.
\end{definition}
レゾルベントを用いると,解は$bm{y}(x) = R(x,x_0)\bm{c}$となる.

\begin{definition}[ロンスキー行列・ロンスキアン]
    $n$つの$n$次元ベクトル値関数$\bm{y}_1(x),\ldots,\bm{y}_n(x)$に対して
    \[ W(x) \coloneqq (\bm{y}_1(x),\ldots,\bm{y}_n(x))\]
    を\textbf{ロンスキー行列}といい,その行列式すなわち$\det (\bm{y}_1(x),\ldots,\bm{y}_n(x))$を\textbf{ロンスキアン}という.
\end{definition}

\begin{theorem}[Liouvilleの定理]
    $\bm{y}_k(x)$は$\bm{y}'_k = A\bm{y}_k$をみたすとする.このとき,以下がなりたつ:
    \[ W(x) = W(x_0) \exp \left(\int_{x_0}^x \tr A(x) \; dx\right).\]
\end{theorem}
一般に$\exp(\cdots) \neq 0$であるから,次のことがいえる:
\begin{note*}
    ある$x_0$で$W(x_0) \neq 0.$ $\iff$ $W(x) \neq 0.$
\end{note*}

$\{\bm{y}_1,\ldots,\bm{y}_n\}$が基本解であるとする.
\[ Y(x_0) = (\bm{y}_1(x_0) \; \cdots \; \bm{y}_n(x_0)) = (\bm{v}_1 \; \cdots \; \bm{v}_n) = V(x)\]
であり,$W(x_0) = \det Y(x_0) = \det V \neq 0$とLiouvilleの定理から,$Y(x)$の逆行列は常に存在することがわかる.

\subsection{(H)}
