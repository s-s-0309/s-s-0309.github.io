\section{色々な常微分方程式,逐次近似法} %第4回
\subsection{常微分方程式}
次のような常微分方程式を考えよう:
\begin{example}[1階の常微分方程式]
    $y'(x) - y(x) = 0$において,$y=e^x$は解である.
\end{example}

\begin{example}[2階の常微分方程式] \label{ex:2order-ode}
    $y''(x) - y(x) = 0$において,$y=\sin x, \; y=\cos x$は解である.
\end{example}

\begin{example}[$n$階の常微分方程式]
    $f(x)$を0を含む区間で定義された関数とする.$y^{(n)} = f(x)$については,$n$回積分すればもとまりそうである.では実際に積分してみよう.
    \begin{align*}
        \int_{0}^{x} y^{(n)}(x) \; dx &= \int_{0}^{x} f(x) \; dx \\
        y^{(n-1)} &= \int_{0}^{x} f(x) \; dx + c_1 \quad (c_1 \coloneqq f^{(n-1)}(0)).
    \end{align*}
    同様にして
    \begin{align*}
        y^{(n-2)}(x) &= \int_{0}^{x} \left( \int_{0}^{x_{n-1}} f(x_1) \; dx + c_1 \right) \; dx + c_2 \\
        &= \int_{0}^{x} \int_{0}^{x_{n-1}} f(x_n) \; dx_n \; dx_{n-1} + c_1x + c_2.
    \end{align*}
    これをずっと続けると,次のような結果を得る:
    \begin{equation}
        y(x) = \int_{0}^{x} \int_{0}^{x_1} \cdots \int_{0}^{x_{n-1}} f(x_n) \; dx_n \; dx_{n-1} \cdots dx_1 + \sum_{k=1}^{n} \frac{c_k}{(n-k)!}x^{n-k}.
    \end{equation}
    ただし,$c_1,\ldots,c_n$は任意定数である.

    
\end{example}

一般に,次のことが言える:
\begin{itemize}
    \item $n$階の常微分方程式は$n$つの任意定数を含む $\longrightarrow$ 一般解,
    \item 定数に特別な値を代入して得られる解 $\longrightarrow$ 特殊解,
    \item どんな値を代入しても得られない解 $\longrightarrow$ 特異解.
\end{itemize}

\begin{example}
    \ref{ex:2order-ode}において,$y=c_1\cos x + c_2 \sin x$は$y'' + y= 0$をみたす.これは\textbf{一般解}.ここで,$y_1=\cos x + 2\sin x$について,${y_1}''+y_1 = 0$をみたすから,$y_1$は解である.加えて,$y_1$は$(c_1,c_2)=(1,2)$としたものだから,これは特殊解である.

    $\tilde{y} \coloneqq c_1(\cos x + \sin x) + c_2 \sin x$も,任意の$c_1,c_2$について$y''+y=0$をみたすから,これは一般解である.このとき,$y_1=\cos x + 2\sin x$は$(c_1,c_2)=(1,1)$としたときの特殊解である.
\end{example}

\begin{remark}
    一般解のによって,$c_1,c_2$の組は変わる.
\end{remark}

\begin{homework*}
    $y \coloneqq c_1 e^{ix} + c_2 e^{-ix}$は$y''+y=0$の一般解となることを示せ.
\end{homework*}

\begin{example}
    $y'=xy + xy^2$を考える.このとき,$y_1=\frac{ce^{x^2/2}}{1-ce^{x^2/2}}, \; y_2=\frac{e^{x^2/2}}{c-e^{x^2/2}}$は解である.

    実際,これらを代入してなりたつことを示せばよい.だが,これはガリガリ計算したくない人にとっては苦痛である.そこで,次に見せるような方法だと,それはある程度解消されるかもしれない.

    まず,次がなりたつことは大丈夫だろう:
    \begin{align*}
        y_2 &= \frac{e^{x^2/2}}{c-e^{x^2/2}} \\
        (c-e^{x^2/2})y_2 &= e^{x^2/2}. \\
    \end{align*}

    次に,この両辺を微分すると,以下のような結果を得る:
    \begin{align*}
        -xe^{x^2/2}y_2 + (c-e^{x^2/2}){y_2}' &= xe^{x^2/2} \\
        (c-e^{x^2/2}){y_2}' &= xe^{x^2/2}(1+y_2) \\
        {y_2}' &= \frac{e^{x^2/2}}{c-e^{x^2/2}}(x+xy_2) \\
        {y_2}' &= xy_2 + x{y_2}^2.
    \end{align*}

    さて,$y \equiv 0, \; 1$は解となるのだが,一般解$y_1, \; y_2$とどのように関係しているかをみよう.

    $y_1$について
    \begin{itemize}
        \item $y \equiv 0$は$c=0$のときの特殊解である.
        \item $y \equiv -1$は特異解である.
    \end{itemize}
    $y_2$について
    \begin{itemize}
        \item $y \equiv 0$は特異解である.
        \item $y \equiv -1$は$c=0$のときの特殊解である.
    \end{itemize}
\end{example}

\begin{remark}
    特殊解or特異解は,対応する一般解とセットで考える.
\end{remark}

\subsection{逐次近似法}
\begin{example}
    $y'(x)-y(x)=0$を\textbf{逐次近似法}でとく(これをPicardの方法ともいう).ただし,$x=0$のとき$y(0)=c$とする.
    この常微分方程式を次のように記そう:
    \begin{equation}
        \begin{cases}
            y'(x)-y(x)=0, \\
            y(0)=c.
        \end{cases}
    \end{equation}

    これに対応する積分方程式は以下になる:
    \begin{align*}
        y(x)-y(0) &= \int_0^x y(x) \; dx \\
        y(x) &= \int_0^x y(x) \; dx + c.
    \end{align*}
    
    さて,関数列$\{y_k(x)\}$を次で定める.これは漸化式である.
    \begin{equation}
        \begin{cases}
            y_0(x) = c, \\
            y_k(x) = \int_{0}^{x} y_{k-1}(x) \; dx + c.
        \end{cases}
    \end{equation}

    具体的に計算してみよう.
    \begin{align*}
        y_1(x) &= c + \int_{0}^{x} y_0(x) \; dx = c(1+x), \\
        y_2(x) &= c + \int_{0}^{x} y_1(x) \; dx = c(1+x+\frac{1}{2}x^2), \\
        y_3(x) &= c + \int_{0}^{x} y_2(x) \; dx = c(1+x+\frac{1}{2}x^2+\frac{1}{3!}x^3).
    \end{align*}

    よって
    \begin{equation}
        y_k(x) = c(1+x+\frac{1}{2}x^2+\cdots+\frac{1}{k!}x^k) = \sum_{j=0}^{k} \frac{x^j}{j!}
    \end{equation}
    と推測できる.
\end{example}

\begin{note*}[この推測は正しいの?]
    この推測が正しいことは,帰納法を用いて示せる.実際,$k$で正しいとして
    \begin{align*}
        y_{k+1}(x) &= c + \int_{0}^{x} c\sum_{j=0}^{k} \frac{x^j}{j!} \; dx \\
        &= c + c \sum_{j=o}^{k} \frac{x^{j+1}}{(j+1)!} \\
        &= c \sum_{j=0}^{k} \frac{x^j}{j!}.
    \end{align*}
\end{note*}

$k \to \infty$とすれば,次を得る:
\begin{equation}
    y_{\infty}(x) = c \sum_{j=0}^{\infty} \frac{x^j}{j!} = ce^x.
\end{equation}

このようにして,$y=ce^x$を構成できる.\footnote{この方法を\ref{book:takahashi}で見たとき感動したんですけど,皆さんも感動しませんでしたか?}

\begin{example}
    $y=c_1\cos x + c_2\sin x$は次の微分方程式を満たす:
    \begin{equation}
    \begin{cases}
        y''(x)+y(x) = 0, \\
        y(x) = c_1, \\
        y'(x) = c_2.
    \end{cases}
    \end{equation}

    $y'(x)=z(x)$とおく.$z'=y''=-y$であり,$z(0)=y'(0)=c_2$である.これらを踏まえて,問題の常微分方程式は
    \begin{equation}
        \begin{cases}
            y'(x)= z(x), \\
            y(0) = c_1, \\
            z'(x) = -y, \\
            z(0) = c_2.
        \end{cases}
    \end{equation}
    という,$y$と$z$の1階の常微分方程式系で書ける.

    これに対応する積分方程式は
    \begin{equation}
        \begin{cases}
            y(x) = c_1 + \int_{0}^{x} z(x) \; dx, \\
            z(x) = c_2 - \int_{0}^{x} y(x) \; dx
        \end{cases}
    \end{equation}
    となる.関数列$\{y_k\}, \{z_k\}$を以下で定める:
    \begin{equation}
        \begin{cases}
            y_k = c_1 + \int_{0}^{x} z_{k-1} \; dx, \quad y_0 = c_1, \\
            z_k = c_2 - \int_{0}^{x} y_{k-1} \; dx, \quad z_0 = c_2.
        \end{cases}
    \end{equation}
    試しに$k=1$だけ計算しよう.すると,
    \begin{align*}
        y_1 &= c_1 + \int_{0}^{x} z_0 \; dx = c_1 + c_2 x, \\
        z_1 &= c_2 - \int_{0}^{x} y_0 \; dx = c_2 - c_1 x
    \end{align*}
    となる.これを繰り返すと,以下のような結果を得る:
    \begin{equation}
        y_k(x) = c_1 \sum_{j=0}^{[\frac{k}{2}]} \frac{(-1)^j}{(2j)!} x^j + c_2 \sum_{j=0}^{[\frac{k-1}{2}]} \frac{(-1)^j}{(2j+1)!} x^j.
    \end{equation}
    $k \to \infty$とすると
    \begin{align}
        y_{\infty} &= c_1 \sum_{j=0}^{\infty} \frac{(-1)^j}{(2j)!} x^j + c_2 \sum_{j=0}^{\infty} \\
        &= c_1 \cos x + c_2 \sin x.
    \end{align}
\end{example}

\begin{note*}[ガウス記号]
    $\sum$の上に$[\frac{k}{2}], \; [\frac{k-1}{2}]$という記号が出てきている.これは\textbf{ガウス記号}と呼ばれるものである.$[\frac{k}{2}]$の値は,\textbf{$\frac{k}{2}$を超えない最大の整数}である.

    つまり,$[a]$はある整数$n$で$n \leq a <n+1$となるもののことである.
    
    ガウス記号の例をみよう.$[3.4]$であれば,$[3.4]=3$となる.$[-3.4]$であれば$[-3.4]=4$となる.$[-3.4]=3$だと,$-3$は$-3.4$を超える整数であるため,不適である.
\end{note*}
さて,これが3階,4階と階数が増えたらどうなるだろうか.それについては,各自でぼんやりと考えればよい(とわたしは思う).

とにかく,\textbf{構成できるという事実が大切}ということを,実感してほしい.