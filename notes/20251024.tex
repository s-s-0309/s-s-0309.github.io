\documentclass[dvipdfmx,11pt]{jsarticle}
\usepackage{/Users/shosaitou/Documents/TeXShop/tex-template/mystyle}

\begin{document}

\subsection*{指摘(笠井先生)}
\begin{itemize}
    \item 「条件」とは命題のことなのか否か.冒頭では$x \in A$とパラメータ依存だったが,のちに導入した条件はそうではない.そういったものの区別はどうするのか\\
    $\longrightarrow$ 条件と書くよりも「命題」と書いた方がよい.
    \item 「乗法」のイントネーションがおかしい.法の方が強いのは「情報」.前にアクセントをつけるべき.
    \item 実数体$\mathbb{R}$の和について,0と逆元$-a$の一意性,積について1と逆元$a^{-1}$があるが,これらがそれぞれ一意に存在するということはかなり大事な結果である.だから,一意性のチェックはすぐ通り過ぎるのはちょっと勿体無い.のちほど導入する差$b-a$と商$b/a (a \neq 0)$は,上述した一意性によって保証されているわけであるから,やはり避けられない.
    \item 加法と乗法については,規則がそれぞれ4つずつあるわけだが,それらをみんなに確認しておくことは重要である.自分だけが知っているようではダメ.
    \item 次のような変形は\textbf{普通はしない}:
        \begin{align*}
            ax + b &= ax' + b \\
            ax + b + (-b) &= ax' + b + (-b) \\
            ax + 0 &= ax' + 0 \\
            ax &= ax' \\
            a^{-1} ax &= a^{-1} ax' \\
            x &= x'.
        \end{align*}
        これは同値変形だから別に問題はないが,aを行列$A$とした場合はどうであろうか.\textbf{すべて左辺にもってきて$x-x'$が恒等的に0であることから}$x=x'$と主張するべきである.
\end{itemize}

\end{document}