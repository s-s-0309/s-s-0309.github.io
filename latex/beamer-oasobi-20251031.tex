\documentclass[dvipdfmx,aspectratio=141]{beamer}
%%%---------beamerの設定---------%%%
\usetheme{Madrid} %select: Madrid, Luebeck, metropolis, Frankfurt, Copenhagen, Dresden, etc.
%\usetheme[block=fill,progressbar=foot,numbering=fraction]{metropolis} 
%上は,themeをmetropolisにするときに使う.この設定にすると,NKGW先生らしいデザインになる.

\usecolortheme{orchid}
\usefonttheme{default} %select: default, professionalfonts, serif, structurebold
\useinnertheme{default} %select: default, circles, rectangles, rounded
\useoutertheme{default} %select:default, miniframes, infolines, split, smoothbars
\usepackage{atbegshi} %しおりの文字化け解消
\ifnum 42146=\euc"A4A2
\AtBeginShipoutFirst{\special{pdf:tounicode EUC-UCS2}}
\else
\AtBeginShipoutFirst{\special{pdf:tounicode 90ms-RKSJ-UCS2}}
\fi
%\setbeamercolor{title}{fg=structure, bg=} %title color
%\setbeamercolor{frametitle}{fg=structure, bg=} %frame title color
%\setbeamertemplate{footline}[frame number] %screen slide number

\setbeamertemplate{navigation symbols}{} %変なナビゲーション記号を消す
\setbeamertemplate{theorems}[numbered] %定理環境の番号を順に表示
%日本語で定理環境を示したいときは,ここに[1]を追加

%section毎に目次を表示
\AtBeginSection[]{
\begin{frame}{次のセクション}
  \tableofcontents[currentsection]
\end{frame}}

%%%%%%%%%%%%%%%%%%%%%%%%%%%%%%%%%%%%%%%%%%%%%%%%%%%%%%%%%%%%%%%%%%%%%%%%%%%%%%%%%%%%%%%%%%%%%%%

%%%===和文用(とりあえずぷりアンブルに追加しておこう)===%%%
\renewcommand{\kanjifamilydefault}{\gtdefault} % 和文既定をゴシックに変更
\usepackage{minijs}
\usepackage[deluxe]{otf}
\usepackage[noalphabet]{pxchfon}

\usepackage{framed}

\usepackage{graphicx}
%\usepackage[dvipsnames]{xcolor}
\usepackage{comment}
\usepackage{ulem}


%%%===数式===%%%
\usepackage{amsmath,amssymb,amsthm}
\usepackage{mathrsfs}
\usepackage{bm}


%%%===ハイパーリンク===%%%
\usepackage{hyperref}
\usepackage{pxjahyper}



%%%===title===%%%
\title{Beamerを使ってみよう}
%\subtitle{}
\author[斉藤尚]{斉藤尚} %; Sho Saitou
\institute[FU]{福島大学 共生システム理工学類} %; Fukushima University Faculty of Symbiotic System Sciences
\date{last updated: \today}
%%%===logo===%%%


%%%===本文開始===%%%
\begin{document}
\maketitle

\begin{frame}{目次}
    \tableofcontents
\end{frame}

\section{数式を打ってみよう}
\begin{frame}{ブロック環境の紹介}
  \begin{block}{普通のブロック}
    普通のブロック.何をかいてもいいよ.
  \end{block}
  \begin{alertblock}{アラートブロック}
    alertブロック.何をかいてもいいよ.
  \end{alertblock}
  \begin{exampleblock}{exampleブロック}
    exampleブロック.何をかいてもいいよ.
  \end{exampleblock}
\end{frame}

\section{定理環境を使ってみよう}
\begin{frame}{定理環境の紹介}
  \begin{definition}[数列の収束]
    実数列$\{a_n\}$が$\alpha \in \mathbb{R}$に収束するとは,次がなりたつときをいう:
    \[ \forall \varepsilon>0, \exists N \in \mathbb{N} \quad \text{s.t.} \quad n \geq N \implies |a_n - \alpha|<\varepsilon.\]
    また,これを$\lim_{n \to \infty}a_n = \alpha$とか$a_n \to \alpha \; (n \to \infty)$とかく.
  \end{definition}
  \begin{theorem}
    $\{a_n\}, \{b_n\}$を数列とし,$\lim_{n \to \infty} a_n = \alpha, \; \lim_{n \to \infty} b_n = \beta$であるとする.このとき,次がなりたつ:
    \[ \lim_{n \to \infty} (a_n + b_n) = \alpha + \beta.\]
  \end{theorem}
\end{frame}

\section{微分方程式のスライドを作ってみよう}
\begin{frame}{微分方程式とは...}
    そもそも微分方程式とはなんだろうか.当たり前ではあるが,言葉の意味を確認しておこう.
    \begin{definition}
        \textbf{微分方程式}とは,未知関数の微分を含んだ方程式のことである.
    \end{definition}
    次のスライドで,微分方程式の具体例を見てみよう.
\end{frame}

\begin{frame}{具体例}
    \begin{example}
        $dx/dt = x$は微分方程式.ちなみにこれを解くと$x=Ce^t$($C$は任意定数)となる.最も基本的な微分方程式.高校生でもわかる.
    \end{example}
    \begin{example}
        $d^2x/dt^2 = -x$は微分方程式.ちなみにこれを解くと$x = C_1\cos x + C_2\sin x$($C_1,C_2$は任意)となる.単振動でよくみるだろう.
    \end{example}
    \begin{example}
        次の微分方程式は\textbf{熱方程式}と呼ばれる.
        \[ \frac{\partial u}{\partial t} = \Delta u.\]
        通常初期値,境界値を設定するが,今回は省いている.
    \end{example}
\end{frame}



\end{document}

https://qiita.com/birdwatcher/items/5dd8a5f453bdc0c6940e
http://xyoshiki.web.fc2.com/tex/beamer.html#usefonttheme

[1]
\uselanguage{japanese}
\languagepath{japanese}
\deftranslation[to=japanese]{Theorem}{定理}
\deftranslation[to=japanese]{Lemma}{補題}
\deftranslation[to=japanese]{Example}{例}
\deftranslation[to=japanese]{Examples}{例}
\deftranslation[to=japanese]{Definition}{定義}
\deftranslation[to=japanese]{Definitions}{定義}
\deftranslation[to=japanese]{Problem}{問題}
\deftranslation[to=japanese]{Solution}{解}
\deftranslation[to=japanese]{Fact}{事実}
\deftranslation[to=japanese]{Proof}{証明}
\def\proofname{証明}