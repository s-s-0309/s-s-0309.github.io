\section{完全微分方程式} %第9回
次のような方程式を考えよう:
\begin{equation}
    P(x,y) + Q(x,y)y' = 0. \label{eq:perfect-ode}
\end{equation}
ただし,$P,\;Q$は$x,y$の2変数関数で既知とする.$y'=dy/dx$であったから,形式的には次のようにも捉えられる:
\begin{equation}
    P(x,y) \; dx + Q(x,y) \; dy = 0. \label{eq:perfect-ode-2}
\end{equation}

左辺が,ある2変数関数$u(x,y)$の全微分
\[ du = \frac{\partial }{\partial x}dx + \frac{\partial }{\partial y}dy \]
と表現されたとき,\eqref{eq:perfect-ode}または\eqref{eq:perfect-ode-2}を\textbf{完全微分方程式}という.いま,$du \equiv 0$の解は$u(x,y) \equiv C$となる.ここで,$C$は定数である.

さて質問.
\begin{question*}
    $u(x,y) \equiv C$の解は,\eqref{eq:perfect-ode}または\eqref{eq:perfect-ode-2}をみたすだろうか.
\end{question*}
\begin{proof}[\textbf{答え}]
    正解は\textbf{かける}.実際,以下のようにしてわかる.

    $u(x,y) \equiv C$をみたす$y$は$y=f(x)$とかけたとする.つまり,$u(x,f(x))=0$がなりたつ.両辺を$x$で微分すると
    \begin{align*}
        \frac{\partial u}{\partial x} \frac{\partial x}{\partial x} + \frac{\partial y}{\partial y} \frac{\partial y}{\partial x} &= 0 \\
        \ie \quad u_x + u_y y' &= 0.
    \end{align*}
\end{proof}

\begin{theorem}
    $P(x,y), \; Q(x,y)$は微分できるとする.このとき,次の(1),(2)は同値:
    \begin{enumerate}
        \item \eqref{eq:perfect-ode}は完全微分方程式である.
        \item $\partial P/\partial y = \partial Q/\partial x.$
    \end{enumerate}
    さらに,完全形であるとき,解は次のように表される:
    \[ \int_{x_0}^{x} P(x,y) \; dx + \int_{y_0}^{y} Q(\textcolor{red}{x_0}, y) \; dy.\]
\end{theorem}
\begin{proof}
    ($\Rightarrow$) 完全形$\partial u/\partial x = P(x,y), \; \partial u/\partial y = Q(x,y)$である.それぞれ$y,x$で偏微分すると
    \[ \frac{\partial^2 u}{\partial y \partial x} = \frac{\partial P}{\partial y}, \quad \frac{\partial^2 u}{\partial x \partial y} = \frac{\partial Q}{\partial x}\]
    より,$\partial P/\partial y = \partial Q/\partial x$である.

    ($\Leftarrow$) $u(x,y) = \int_{x_0}^{x} P(x,y) \; dx + \int_{y_0}^{y} Q(x_0,y) \; dy$とおく.微分積分学の基本定理より
    \[ \frac{\partial u}{\partial x} = \frac{\partial}{\partial x} \int_{x_0}^{x} P(x,y) \; dx = P(x,y).\]
    同様にして
    \begin{align*}
        \frac{\partial u}{\partial y} &= \frac{\partial}{\partial y} + Q(x_0,y) \\
        &= \int_{x_0}^{x} \frac{\partial P}{\partial y} \; dx + Q(x_0,y) \\
        &= \int_{x_0}^{x} \frac{\partial Q}{\partial x} \; dx + Q(x_0,y) \\
        &= \left[Q(x,y)\right]_{x_0}^x + Q(x_0,y) = Q(x,y).
    \end{align*}
    よって,$\partial P/\partial y = \partial Q/\partial x$となり,示された.
\end{proof}

今用意した$u$について,$u \equiv c$が解,すなわち
\[ \int_{x_0}^{x} P(x,y) \; dx + \int_{y_0}^{y} Q(\textcolor{red}{x_0}, y) \; dy = c\]
は解となる.

\begin{example}
    $y' = (x - y\cos x)/\sin x$を考える.この式は
    \[ (x - y\cos x) \; dx + (-\sin x) \; dy = 0\]
    と形式的に変形できる.$P(x,y)=x - y\cos x, \; Q(x,y)=-\sin x$であるから,$\partial P/\partial y = -\cos x, \; \partial Q/\partial x = -\cos x$であるから,これは完全微分方程式である.

    したがって解は$\int_{x_0}^{x} (x-y\cos x) \; dx + \int_{y_0}^{y} (-\sin x_0) \; dy = c$となる.特に,$x_0=y_0=0$となると
    \[ \int_0^x (x-y\cos x) \; dx = \frac{x^2}{2} - y\sin x = c\]
    が解となる.
\end{example}

\begin{example}
    $(2x-2y+3) \; dx + (-2x+4y+1) \; dy = 0$を解く.

    $P(x,y) \coloneqq 2x-2y+3, \; Q(x,y) \coloneqq -2x+4y+1$とおくと,$\partial P/\partial y = -2, \; \partial Q/\partial x = -2$であるから,これは完全形である.
\end{example}
\begin{homework*}
    上の例の解を求めよ.
\end{homework*}

今まで紹介した例では,たまたま微分方程式が完全形となったが,そうではないことももちろんある.しかし,そのようなときでも,適切な関数$\lambda(x,y)$をかけて
\[ \lambda(x,y)P(x,y) + \lambda(x,y)Q(x,y)y' = 0\]
が完全形となることがある.このとき,$\lambda$を\textbf{積分因子}という.

\begin{example}[$1/y^2$をかける]
    $y-xy'=0$を考える(これは変数分離形,同次形,線形でもある).$P(x,y) \coloneqq y, \; Q(x,y) \coloneqq -x$とおく.$\partial P/\partial y \neq \partial Q/\partial x$であるから,これは完全形ではない.

    さて,両辺に$y^{-2}$をかけよう.
    \[ \frac{1}{y^2}(y \; dx - x \; dy) = \frac{1}{y} \; dx - \frac{x}{y} \; dy = du.\]
    ここで,$u \coloneqq y/x$とおいた.上式は$u$の全微分であることに注意しよう.したがって,$x=cy$が解である.
\end{example}

\begin{example}[$1/xy$をかける]
    両辺に$1/xy$をかけよう.
    \[ \frac{1}{xy}(y \; dx - x \; dy) = \frac{1}{x} \; dx - \frac{1}{y} \; dy = d\left(\log|\frac{y}{x}|\right).\]
    したがって,$\log|y/x|=c$が解である.
\end{example}

積分因子として$1/y^2$をかけようが$1/xy$をかけようが,得られた結果は変わらない.\textbf{積分因子は1つに決まらないのだ}.一般に,次のことが知られている.

\begin{remark}
    1階の常微分方程式では,積分因子が存在することが知られている.
\end{remark}

\begin{example} \label{example:homework-10-1}
    $xy^2-y^3+(1-xy^2)y'=0$を考える.解を求めるまでは,宿題とする.
\end{example}

\begin{homework*}
    次の手順にしたがって,例\ref{example:homework-10-1}を解け.$P(x,y) \coloneqq xy^2-y^3, \; Q(x,y) \coloneqq 1-xy^2$とする.
    \begin{enumerate}
        \item 完全微分方程式ではないことを示せ.
        \item $\lambda(x,y) \coloneqq x^my^m, \tilde{P}(x,y) \coloneqq (xy^2-y^3)P(x,y), \tilde{Q}(x,y) \coloneqq (1-xy^2)Q(x,y)$とおく.このとき$\partial \tilde{P}/\partial y, \; \partial \tilde{Q}/\partial x$を求めよ.
        \item $\tilde{P}(x,y),\tilde{Q}(x,y)$を踏まえて,完全系であるための必要十分条件を述べよ.
        \item 積分因子$\lambda(x,y)$を求めよ.
    \end{enumerate}
\end{homework*}


