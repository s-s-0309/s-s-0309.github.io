\section{微分演算子(その1)}
次のような定数係数の線形微分方程式を考えよう:
\begin{align}
    y^{(n)}(x) + a_1y^{(n-1)}(x) + \cdots +a_{n-1}y' + a_ny(x) &= f(x), \label{eq:I-13} \\
    y^{(n)}(x) + a_1y^{(n-1)}(x) + \cdots +a_{n-1}y' + a_ny(x) &= 0. \label{eq:H-13}
\end{align}
ただし,$y$は未知関数,$f(x)$は既知であるとする.

ここで,$y' = \frac{d}{dx}y$を次のように書く:
\[ y' \coloneqq Dy \quad \ie D \coloneqq \frac{d}{dx}.\]
2階微分,3階微分,$k$階微分については以下のように書く::
\[ \frac{d^2}{dx^2} = D^2, \; \frac{d^3}{dx^3} = D^3, \; \cdots, \; \frac{d^n}{dx^n} = D^n.\]

これを踏まえると,\eqref{eq:I-13}は次のように書くことができる:
\[ (D^n+a_1D^{n-1}+\cdots+a_{n-1}D+a_n)y = f(x).\]
この$D$を\textbf{微分演算子}とよぶ.

\begin{example}
    $a,b$を定数として,$(D-a)(D-b)y$を計算しよう.
    \begin{align*}
        (D-a)\{(D-b)y\} &= (D-a)(Dy-by) \\
        &= D(y'-by) - a(y'-by) \\
        &= (y''-by') - ay' + aby \\
        &= y'' -(a+b)y' + aby \\
        &= (D^2-(a+b)D+ab)y.
    \end{align*}
    同様にして$(D-b)(D-a)y=(D^2-(b+a)D+ab)y$が導かれる.
\end{example}

\begin{remark}
    $(D-a)$と$(D-b)$の作用の順番は交換可能であり
    \[ (D-a)(D-b)y = (D-b)(D-a)y = (D^2-(a+b)D+ab)y\]
    となり,代数的に扱うことができる.
\end{remark}

微分演算子$D$に対して,sの逆演算$D^{-1}$を
\[ D^{-1} f\coloneqq \int f(x) \; dx\]
で定義する.

\begin{note*}
    \begin{align*}
        D^{-1}(Df) &= f + C, \\
        D(D^{-1}f) &= f
    \end{align*}
    となる.ただし,$C$は定数である.このようにして,$D^{-1}D, DD^{-1}$は少し異なる.ちなみに,$DD^{-1}$を\textbf{恒等演算子}と呼ぶこともある.
\end{note*}

さて,微分演算子$D$の導入により,我々\eqref{eq:I-13}を
\[ (D^n+a_1D^{n-1}+\cdots+a_{n-1}D+a_n)y = f(x)\]
のように書くことに成功した.$L(D) \coloneqq D^n+a_1D^{n-1}+\cdots+a_{n-1}D+a_n$とおこう.\eqref{eq:I-13}は$L(D)=f(x)$のように書き直すことができる.$L(D)^{-1}$が上手に定義できれば,解$y$がわかりそうである.

\begin{example}
    $D^2y = f$について,$(D^2)^{-1}f \coloneqq \iint f(x) \; dxdx$と定義すればよい.以後,$(D^k)^{-1}$を$D^{-k}$と表すこととする.

    ちなみに,$\iint$は$x$について2回積分してくださいということを意味していて,重積分の意味ではないことに注意されたい.
\end{example}

\begin{example}
    $(D-a)y=f$を考える.変形すると$y'-ay=f$となり,これは線形である.両辺に$e^{-ax}$をかけると
    \begin{align*}
        e^{-ax}y' -e^{-ax}y &= e^{-ax}f \\
        (e^{-ax})' &= e^{-ax}f(x).
    \end{align*}
    したがって
    \begin{align*}
        e^{-ax}y &= \int e^{-ax} f(x) \; dx \\
        y &= e^{ax} \int ^{-ax} f(x) \; dx.
    \end{align*}
    この考察から,我々は$(D-a)^{-1}$を次で定義する:
    \[ (D-a)^{-1} \coloneqq e^{ax} \int e^{-ax}f(x) \; dx.\]
\end{example}

\begin{example}
    $(D-a)^2y = f(x)$を考える.$((D-a)^2)^{-1}f(x)$が知りたい.先ほどの議論より$((D-a)^2)^{-1} = (D-a)^{-2}$とかける.$z(x) \coloneqq (D-a)y$とおくと,$(D-a)z=f(x)$となる.したがって
    \[ z(x) = e^{ax} \int e^{-ax} z(x) \; dx.\]
    さらに
    \begin{align*}
        y &= (D-a)^{-1} z(x) \\
        &= e^{ax} \int e^{-ax} z(x) \; dx \\
        &= e^{ax} \int e^{-ax} \left(e^{ax} \int e^{-ax} f(x) \; dx\right) \; dx \\
        &= e^{ax} \iint e^{-ax} f(x) \; dx.
    \end{align*}
    先ほども述べたが,$\iint$は2重積分の意味ではない.この考察から
    \[ (D-a)^{-2} \coloneqq e^{ax} \iint e^{-ax} f(x) \; dx\]
    と定義すればよい.

    同様に,$(D-a)^{-k}y=f$については
    \[ (D-a)^{-k} \coloneqq e^{ax} \iint \cdots \int e^{-ax} f(x) \; dx\]
    と定義すればよい.
\end{example}

\begin{example}
    $(D-a)(D-b)y=f(x)$を考える.$((D-a)(D-b))^{-1}f$が知りたい.$z(x) \coloneqq (D-b)y$とおくと,$y=(D-b)^{-1}z(x)$となり,$y=(D-b)^{-1}z(x)$を得る.したがって,$y=(D-b)^{-1}(D-a)^{-1}f$となる.よって,次の関係を得る:
    \[ ((D-a)(D-b))^{-1} = (D-b)^{-1}(D-a)^{-1}.\]
    一方,$(D-a)(D-b)=(D-b)(D-a)$であるから
    \[ (D-b)^{-1}(D-a)^{-1} = (D-a)^{-1}(D-b)^{-1}\]
    となる.
\end{example}

以上をまとめ,$L(D)$を因数分解すると
\[ L(\lambda) = (\lambda - \lambda_1)^{n_1} ((\lambda - \lambda_2)^{n_2}) \cdots (\lambda - \lambda_r)^{n_r}\]
となる.ただし,$n_k$は$\lambda_k$の重複度,$i \neq j$のとき$\lambda_i \neq \lambda_j$,$n_1+n_2+\cdots+n_r=n$とする.このとき,$(D-\lambda_1)^{-n_1} (D-\lambda_2)^{-n_2} \cdots (D-\lambda_r)^{-n_r}f$は$L(D)y=f$という\eqref{eq:I-13}の1つの解となる.$L(D)y=f$の一般解は,基本解の一次結合にこれを加えたものになる.

$L(D)y=0$の基本解は,$(D-\lambda_j)^{-n_j}y=0$をみたすものを探せばよい.このとき,$\lambda_j$つの以下の解がでてくる:
\[ e^{\lambda_j x}, \; xe^{\lambda_j x}, \; x^2e^{\lambda_j x}, \; \ldots, \; x^{\lambda_j-1}e^{\lambda_j x}.\]

$L(D)y=0$の基本解は,次で表せる:
\[ y(x) = \sum_{j=1}^{r} \sum_{k}^{n_j-1} C_{jk} x^k e^{\lambda_j x}.\]

\begin{definition}
    \eqref{eq:I-13}:$L(D)y=f$の一般解は以下で書ける:
    \[ y = \sum_{j=1}^{r} \sum_{k}^{n_j-1} C_{jk} x^k e^{\lambda_j x} + (D-\lambda_1)^{-n_1} (D-\lambda_2)^{-n_2} \cdots (D-\lambda_r)^{-n_r}f.\]
    ここで,$(D-\lambda_j)^{-n_j}f$は次のようになる:
    \[ (D-\lambda_j)^{-n_j}f = e^{\lambda_j x} \iint \cdots \int e^{-\lambda_j x} f(x) \; dx.\]
\end{definition}

\begin{example}
    $y'''-y''-y'+y=xe^{2x}$を考える.微分演算子を用いると
    \[ (D^3-D^2-D+1)y = xe^{2x}\]
    と書き直せる.ここで,$L(D) \coloneqq D^3-D^2-D+1$とおくと,特性多項式$L(\lambda)$は$L(\lambda)=\lambda^3-\lambda^2+\lambda+1$となる.これを解くと,$\lambda=-1, 1$を得るから,一般解は次のようになる:
    \[ y(x) = c_1e^{-x} + c_2e^x + c_3xe^x + (D-1)^{-2}(D+1)^{-1}xe^{2x}.\]

    特殊解を求めよう.講義では4つの方法を紹介した.この第13回では,3つの方法を紹介する.
\end{example}

\begin{description}
    \item[(1) 定義通りにとく] まず
    \begin{align*}
        (D+1)^{-1}xe^{2x} &= e^{-x} \int e^x xe^{2x} \; dx \\
        &= e^{-x} \int x e^{3x} \; dx \\
        &= e^{-x} \left[\frac{x}{3}e^{3x}\right] - e^{-x} \int \frac{1}{3}e^{3x} \; dx \\
        &= \frac{1}{3}xe^{2x} - \frac{1}{9}e^{2x} \\
        &= \frac{1}{9}(3x-1)e^{2x}.
    \end{align*}
    つぎに
    \begin{align*}
        (D-1)^{-2}(D+1)^{-1}xe^{2x} &= (D-1)^{-2} \left\{\frac{1}{9}(3x-1)e^{2x}\right\} \\
        &= \frac{1}{9}e^x \iint e^{-x} (3x-1) e^{2x} \; dxdx \\
        &= \frac{1}{9}e^x \iint (3x-1)e^x \; dxdx \\
        &= \frac{1}{9}e^x \left(\int [(3x-1)e^x] - 3\int e^x \; dx \right) \\
        &= \frac{1}{9}e^x \int (3x-4)e^x \; dx \\
        &= \frac{1}{9}e^x \left[(3x-4)e^x - 3 \int e^x \; dx\right] \\
        &= \frac{1}{9}e^x \left[(3x-7)e^x\right] \\
        &= \frac{1}{9}(3x-7)e^{2x}.
    \end{align*}
    よって,$(D-1)^{-2}(D+1)^{-1}xe^{2x} = \frac{1}{9}(3x-7)e^{2x}$を得る.

    これは大変!
    
    \item[(2) 推移法則を使う] まず,主張は以下の通りである:
    \begin{theorem}[推移法則]
        $\lambda$を定数とするとき,次がなりたつ:
        \[ L(D)^{-1}(e^{\lambda x}f(x)) = e^{\lambda x} L(D+\lambda)^{-1} f(x).\]
    \end{theorem}
    \begin{align*}
        (D+1)^{-1}xe^{2x} &= e^{2x} L(D+3)^{-1} x \\
        &= e^{2x} \cdot e^{-3x} \int x e^{3x} \; dx \\
        &= e^x \left[\frac{1}{3}xe^{3x} - \frac{1}{3}\int e^{3x} \; dx\right] \\
        &= \frac{1}{9}e^{2x}(3x-1), \\
        (D-1)^{-2}(D+1)^{-1}xe^{2x} &= (D-1)^{-2} \left(\frac{1}{9}e^{2x}(3x-1)\right) \\
        &= e^{2x} (D+1)^{-2} \left(\frac{1}{9}(3x-1)\right) \\
        &= \frac{1}{9}(3x-7)e^{2x}
    \end{align*}
    となり,たしかに(1)の結果と一致する.

    \item[(3) 部分分数分解を用いる] $1/L(\lambda)$が次のように変形できたとしよう:
    \[ \frac{1}{L(\lambda)} = \frac{A_1}{(\lambda-\mu_1)^{n_1}} + \cdots + \frac{A_k}{(\lambda-\mu_k)^{n_k}}.\]
    $L(D)^{-1} = A_1(\lambda-\mu_1)^{n_1} + \cdots + A_k(\lambda-\mu_k)^{n_k}$となる.

    今回の場合,$L(\lambda)=(\lambda-1)^2(\lambda+1)$だから
    \[ L(D)^{-1} = \frac{1}{4}(D+1)^{-1} - \frac{1}{4}(D-1)^{-1} + \frac{1}{2}(D-1)^{-2}\]
    となる.あとは,計算.
\end{description}

\begin{homework*}
    省略された計算過程を自力で補え.
\end{homework*}

次回はMaclaurineの方法を述べる.
